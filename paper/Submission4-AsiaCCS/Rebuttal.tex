\section{Rebuttal For CCS2020}
We would like to thank the reviewers for their comments. We will make an editorial pass to correct all of the typos that the reviewers identified. There are a few substantial misconceptions about the paper which we wish to correct below.

R2: We stress that we do model an attacker who adapts its guessing strategy in response to the DALock mechanism. We introduce the password knapsack problem (see Section 5.4) specifically to model an optimal adaptive attacker and view this as a key contribution of the work. R1 argues that even a small ban list (e.g., top 50 passwords) would effectively mitigate the risk of an offline attack referencing a Microsoft blog. However, an adaptive attacker would simply move on to guess the most popular passwords that have not been banned. The attack would still be quite effective (e.g., passwords 51, 52, … are still quite popular!)

Both reviewers suggested that we consider a range of banlist sizes (e.g., $10^2,10^3,10^4, 10^5$). Increasing the banlist size inconveniences more users when they cannot select their preferred password i.e., banning the top $10^3/10^4/10^5$ passwords would inconvenience roughly 7\%/14\%/24\% of LinkedIn users respectively. We had picked $10^4$ as it seems to strike a reasonable balance between security (flattening the password distribution) and usability. However, we agree that it would be worthwhile to run the simulations with a range of banlist sizes. The updated experiments only require very minor modifications and could be run quickly. R2 claims that the empirical experiments do not show much improvement over traditional k-strike schemes. We disagree. Even when we consider a banlist of size $10^4$ to flatten the distribution DALock still offers a greatly superior trade-off between security and usability. In Figure 7 we see that the same (10,$\hitCountThreshold$=$2^{-11}$)-DALock instantiation outperforms the classical 3-strike mechanism from a security standpoint cutting number of cracked roughly in half --- note that DALock dramatically outperforms the 10-strike mechanism in the same plot. From a usability standpoint (10,$\hitCountThreshold=2^{-11}$)-DALock is dramatically better than 3-strikes and essentially equivalent to 10-strikes.

R1 and R2: Regarding the relationship to StopGuessing (Tian et al.): We certainly agree that there are similar ideas behind StopGuessing and DALock. However, we disagree that the differences are "mostly nuances." As context the work on DALock was ongoing long before the StopGuessing paper was published. As such the DALock mechanism was developed independently of StopGuessing. One can view the two papers as taking complementary/orthogonal approaches i.e., StopGuessing primarily focuses on detecting IP addresses associated with malicious attackers and DALock focuses primarily on detecting attacks against specific accounts.