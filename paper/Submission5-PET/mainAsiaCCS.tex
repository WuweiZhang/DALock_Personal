%%
%% This is file `sample-sigconf.tex',
%% generated with the docstrip utility.
%%
%% The original source files were:
%%
%% samples.dtx  (with options: `sigconf')
%% 
%% IMPORTANT NOTICE:
%% 
%% For the copyright see the source file.
%% 
%% Any modified versions of this file must be renamed
%% with new filenames distinct from sample-sigconf.tex.
%% 
%% For distribution of the original source see the terms
%% for copying and modification in the file samples.dtx.
%% 
%% This generated file may be distributed as long as the
%% original source files, as listed above, are part of the
%% same distribution. (The sources need not necessarily be
%% in the same archive or directory.)
%%
%%
%% Commands for TeXCount
%TC:macro \cite [option:text,text]
%TC:macro \citep [option:text,text]
%TC:macro \citet [option:text,text]
%TC:envir table 0 1
%TC:envir table* 0 1
%TC:envir tabular [ignore] word
%TC:envir displaymath 0 word
%TC:envir math 0 word
%TC:envir comment 0 0
%%
%%
%% The first command in your LaTeX source must be the \documentclass command.
\documentclass[sigconf,anonymous=true]{acmart}
% !TEX root = main.tex

% Packages All the Way

\usepackage{balance,color,amsmath, alltt, xspace, epsfig, algorithm, subfigure, xcolor, multirow, psfrag,mathtools,comment, verbatim,algpseudocode,grffile}					
\usepackage{xurl}

%\usepackage[latin1]{inputenc}
\usepackage{tikz}
\usetikzlibrary{shapes,arrows}

\tikzstyle{decision} = [diamond, draw, fill=blue!20, 
text width=4.5em, text badly centered, node distance=3cm, inner sep=0pt]
\tikzstyle{block} = [rectangle, draw, fill=blue!20, 
text width=5em, text centered, rounded corners]
\tikzstyle{line} = [draw, -latex']

\tikzstyle{cloud} = [draw, ellipse,fill=red!20, node distance=3cm,
minimum height=2em]

 \usepackage{graphicx} 
\usepackage{caption}
\captionsetup[figure]{font=small}
%\usepackage[hyphens]{url}
\usepackage{hyperref}
\hypersetup{breaklinks=true}


% Variable Part
\newcommand{\algoname}{DALock}					%Give a name to the algorithm/theorem


\newcommand{\authnote}[3]{\textcolor{#2}{{\sf (#1's Note: {\sl{#3}})}}}
\newcommand{\wuwei}{\authnote{Wuwei}{green}}
\newcommand{\jeremiah}{\authnote{Jeremiah}{blue}}
\newcommand{\ignore}[1]{}

% Fixed/Constant Part:
%\newtheorem{theorem}{Theorem}
%\newtheorem{definition}{Definition}
\newcommand{\mypara}[1]{\noindent\textbf{#1} \xspace}		% Bold paragraph title
\newcommand{\says}[2]{{\color{blue}{#1 says: }{#2}}\xspace}					% Says 
\DeclarePairedDelimiter{\ceil}{\lceil}{\rceil}								%\ceil	

%---------- Macros for Table (Below)-------------
\newcommand{\PasswordOfU}[1]{pw_{#1}}
\newcommand{\PwOfU}[1]{pw_{#1}}
\newcommand{\PsiOfU}{\Psi_u}
\newcommand{\KOfU}{K_u}
\newcommand{\PwProbEstimator}{\mathsf{Est}}
\newcommand{\AllUser}{\mathcal{U}}
\newcommand{\RankRPassword}[1]{pw_{#1}}
\newcommand{\user}{u}
\newcommand{\Lap}{\mathsf{Lap}\xspace}								% Laplace Noise
\newcommand{\epsLap}[2]{\ensuremath{\mathsf{LAP(\frac{#2}{#1})}}\xspace} %Laplace Noise with #1 privacy budget (\epsilon) and #2 sensitivity
\newcommand{\Adversary}{\ensuremath{\mathcal{A}}}							% Lazy \mathcal{A}
\newcommand{\lazyref}[2]{\textbf{#1}~\ref{#2}}
\newcommand{\ZX}{\ZXCVBN} 			%Lazy fancy ZXCVBN
\newcommand{\loginActivity}[3]{\mathsf{L_{#1}(#2,#3)}}
\newcommand{\ZXCVBN}{\mathsf{ZXCVBN}}
\newcommand{\PCFG}{\mathsf{PCFG}}
\newcommand{\Min}{\mathsf{Min}}
\newcommand{\HashCat}{\mathsf{HashCat}}
\newcommand{\NeuralNet}{\mathsf{NeuralNet}}
\newcommand{\Markov}{\mathsf{Markov}}
\newcommand{\Estimator}{\mathsf{Estimator}}
\newcommand{\EstF}[1]{\textsf{Estimate}(#1)}
\newcommand{\EstimateF}[2]{\textsf{Estimate}(#1,#2)}			%Estimate Frequency of p 
\newcommand{\EstP}[1]{\textsf{p}(#1)}
\newcommand{\EstimateP}[2]{\textsf{p}(#1,#2)}			%Estimate popularity of p 
\newcommand{\MM}{\ensuremath{\mathcal{M}}}							% Lazy \mathcal{A}
\newcommand{\FMPPF}{\ensuremath{\mathsf{FMPPF}}}							% Lazy \mathsf{FMPPF}
\newcommand{\DAB}{\ensuremath{\mathsf{DAB}}}	
\newcommand{\PK}{\ensuremath{\mathsf{PK}}}							% Lazy Password Knapsack
\newcommand{\SampledData}[1]{\mathcal{D}_{{#1}}}
\renewcommand{\Pr}[1]{\ensuremath{\mathsf{Pr} \left[#1\right] }\xspace}			% Fancy Pr
\newcommand{\KPsiDALock}[2]{\ensuremath{(#1, #2)\text{-}\mathsf{DALock}}\xspace}		
\newcommand{\hitCountThreshold}{\Psi}
\newcommand{\hitCountThresholdOfU}[1]{\hitCountThreshold_{#1}}	
\newcommand{\DP}[2]{\textsf{DP}(#1,#2)}
\newcommand{\strikeThreshold}{K}	
\newcommand{\strikeThresholdOfU}[1]{\strikeThreshold_{#1}}
\newcommand{\DALock}{\mathsf{DALock}\xspace}	
\newcommand{\AllPassword}{\mathcal{P}}
\newcommand{\CountSketch}{\mathsf{CS}}
\newcommand{\CountSketchCounter}{\mathsf{CS.T}}
\newcommand{\CountSketchArray}{\mathsf{CS.ARRAY}}
\newcommand{\EstProbOfPw}[2]{\ensuremath{\Estimator_{\mathsf{#2}}}\left(#1\right)}
\newcommand{\TrueP}[1]{\ensuremath{\mathsf{P}\left(#1\right)}}
\newcommand{\TrueF}[1]{\ensuremath{\mathsf{F}\left(#1\right)}}
\newcommand{\TrueFInD}[2]{\ensuremath{\mathsf{F}\left(#1, #2\right)}}
\newcommand{\CSWidth}{w}
\newcommand{\CSDepth}{d}
\newcommand{\TotalFreq}[1]{\ensuremath{\mathsf{TotalFreq}(#1)}}
\newcommand{\Add}[2]{\ensuremath{\mathsf{Add}(#1,#2)}}
\newcommand{\Initialize}[2]{\ensuremath{\mathsf{Initialize}(#1,#2)}}
\newcommand{\HashFunRowD}{\ensuremath{\mathsf{h}_d}}
\newcommand{\HashFunSign}{\ensuremath{\mathsf{h}_{\pm}}}
\newcommand{\hitCountThresholdofUAtT}[2]{\hitCountThresholdOfU{#1}^{#2}}
\newcommand{\strikeThresholdofUAtT}[2]{\strikeThresholdOfU{#1}^{#2}}
%---------- Macros for Table (Above)-------------


\newtheorem{definition}{Definition}
\newtheorem{thm}{Theorem}[section]
\newcommand{\NP}{\mathsf{NP}\xspace}

\newcommand{\FPTAS}{\mathsf{FPTAS}\xspace}
\newcommand{\PTAS}{\mathsf{PTAS}\xspace}
\renewcommand{\P}{\mathsf{P}\xspace}	
\newcommand{\SAT}{\mathsf{SAT}\xspace}	
\newcommand{\TIME}{\mathsf{TIME}\xspace}
\newcommand{\NPC}{\mathsf{NPC}\xspace}	
\newcommand{\myexp}[1]{\ensuremath{e^{#1}}\xspace}						% exp


%---------- Macros for General Setting-------------
\algrenewcommand\algorithmicindent{1.0em}%
\usepackage{tikz}
\usetikzlibrary{matrix,calc,shapes}
\tikzset{
  treenode/.style = {shape=rectangle, rounded corners,
                     draw, anchor=center,
                     text width=5em, align=center,
                     top color=white, bottom color=blue!20,
                     inner sep=1ex},
  decision/.style = {treenode, diamond, inner sep=0pt},
  root/.style     = {treenode, font=\Large, bottom color=red!30},
  env/.style      = {treenode, font=\ttfamily\normalsize},
  finish/.style   = {root, bottom color=green!40},
  dummy/.style    = {circle,draw}
}
\newcommand{\yes}{edge node [above] {yes}}
\newcommand{\no}{edge  node [left]  {no}}




\usepackage{amsfonts}
\usepackage{float}

%%
%% \BibTeX command to typeset BibTeX logo in the docs
\AtBeginDocument{%
  \providecommand\BibTeX{{%
    \normalfont B\kern-0.5em{\scshape i\kern-0.25em b}\kern-0.8em\TeX}}}

%% Rights management information.  This information is sent to you
%% when you complete the rights form.  These commands have SAMPLE
%% values in them; it is your responsibility as an author to replace
%% the commands and values with those provided to you when you
%% complete the rights form.
\setcopyright{acmcopyright}
\copyrightyear{2021}
\acmYear{2022}
\acmDOI{10.1145/1122445.1122456}

%% These commands are for a PROCEEDINGS abstract or paper.
\acmConference[AsiaCCS '22]{AsiaCCS '22: ACM ASIA Conference on Computer and Communications Security }{May 30-- June -3, 2022}{Nagasaki, Japan}
\acmBooktitle{AsiaCCS '22: ACM ASIA Conference on Computer and Communications Security,
May 30-- June -3, 2022, Nagasaki, Japan}
\acmPrice{15.00}
\acmISBN{978-1-4503-XXXX-X/18/06}


%%
%% Submission ID.
%% Use this when submitting an article to a sponsored event. You'll
%% receive a unique submission ID from the organizers
%% of the event, and this ID should be used as the parameter to this command.
%%\acmSubmissionID{123-A56-BU3}

%%
%% The majority of ACM publications use numbered citations and
%% references.  The command \citestyle{authoryear} switches to the
%% "author year" style.
%%
%% If you are preparing content for an event
%% sponsored by ACM SIGGRAPH, you must use the "author year" style of
%% citations and references.
%% Uncommenting
%% the next command will enable that style.
%%\citestyle{acmauthoryear}

%%
%% end of the preamble, start of the body of the document source.
\begin{document}

%%
%% The "title" command has an optional parameter,
%% allowing the author to define a "short title" to be used in page headers.
\title{$\DALock$: Password \underline{D}istribution-\underline{A}ware Throttling}

%%
%% The "author" command and its associated commands are used to define
%% the authors and their affiliations.
%% Of note is the shared affiliation of the first two authors, and the
%% "authornote" and "authornotemark" commands
%% used to denote shared contribution to the research.
\author{Jeremiah Blocki}
\email{jblocki@purdue.edu}
\author{Wuwei Zhang}
\email{zhan1015@purdue.edu}
\affiliation{%
  \institution{Institute for Clarity in Documentation}
  \streetaddress{P.O. Box 1212}
  \city{Dublin}
  \state{Ohio}
  \country{USA}
  \postcode{43017-6221}
}



%%
%% By default, the full list of authors will be used in the page
%% headers. Often, this list is too long, and will overlap
%% other information printed in the page headers. This command allows
%% the author to define a more concise list
%% of authors' names for this purpose.
\renewcommand{\shortauthors}{Blocki and Zhang}

%%
%% The abstract is a short summary of the work to be presented in the
%% article.
\begin{abstract}
\input{abstract}
\end{abstract}

%%
%% The code below is generated by the tool at http://dl.acm.org/ccs.cfm.
%% Please copy and paste the code instead of the example below.
%%
\begin{CCSXML}
	<ccs2012>
	<concept>
	<concept_id>10002978.10003029.10011703</concept_id>
	<concept_desc>Security and privacy~Usability in security and privacy</concept_desc>
	<concept_significance>500</concept_significance>
	</concept>
	<concept>
	<concept_id>10002978.10002991.10002992</concept_id>
	<concept_desc>Security and privacy~Authentication</concept_desc>
	<concept_significance>500</concept_significance>
	</concept>
	</ccs2012>
\end{CCSXML}


\ccsdesc[500]{Security and privacy~Usability in security and privacy}
\ccsdesc[500]{Security and privacy~Authentication}

%%
%% Keywords. The author(s) should pick words that accurately describe
%% the work being presented. Separate the keywords with commas.
\keywords{Authentication Throttling; Password; Dictionary Attack}


%%
%% This command processes the author and affiliation and title
%% information and builds the first part of the formatted document.
\maketitle


% !TEX root = main.tex
	% !TEX root = main.tex

	%\vspace{-0.1in}
\section{Introduction}\label{sec: Introduction}

An online password attacker repeatedly attempts to login to an authentication server submitting a different guess for the target user's password on each attempt. The human tendency to pick weak (``low-entropy'') passwords has been well documented, e.g., ~\cite{SP:Bonneau12}. An untargeted online attacker will typically submit the most popular password choices consistent with the password requirements (e.g., ``Password1''). In contrast, a targeted attacker~\cite{CCS:WZWYH16} might additionally incorporate background knowledge about the specific target user (e.g., birthdate, phone number, anniversary, etc.). To protect users against online attackers, most authentication servers incorporate some form of throttling mechanism. In particular, the $\strikeThreshold$-strikes mechanism temporarily locks a user's account if $\strikeThreshold$-consecutive incorrect passwords are attempted within a predefined time (e.g., $24$ hours). Setting the lockout parameter $\strikeThreshold$ induces a classic security-usability trade-off. Selecting small values of $\strikeThreshold$ (e.g., $\strikeThreshold=3$) provides better protection against online attackers but may result in many unwanted lockouts when an honest user miss-types (or miss-remembers) their password. Selecting a larger value of $\strikeThreshold$ (e.g., $\strikeThreshold=10$) will reduce the unwanted lockout rate but may increase vulnerability to online attacks. 



Bonneau et al.~\cite{SP:BHVS12} considered many proposed replacements for password authentication, finding that all proposals have some drawbacks compared with passwords. For example, passwords are easier to revoke than biometrics. Similarly, hardware tokens are expensive and require users to carry them around. By contrast, passwords are easy to deploy and do not require users to carry anything around. Put simply, we have not found a ``silver bullet'' replacement for passwords. Thus, despite all of their shortcomings (and many attempts to replace them), passwords will likely remain entrenched as the dominant form of authentication on the internet~\cite{PasswordPersistence}. Thus, protecting passwords against online attacks without locking out legitimate users remains a crucial challenge for the foreseeable future~\cite{DuoWeakPassword,DictionaryAttack:Ransomware,DictionaryAttack:Microsoft}. 


One approach to protect users against online guessing attacks is to adopt strict password composition policies to prevent users from selecting weak passwords. However, it has been well documented that users dislike restrictive policies and often respond in predictable ways~\cite{KSKMBCCE:SIGCHI11}. Another defense is to store cookies on the user's device to prove that the next login attempt comes from a known device. Similarly, one can also utilize features such as IP address, geographical location, device, and time of day~\cite{sandhu2005system,gordon2014efficiently,NDSS:FJDBG16} to help distinguish between malicious and benign login attempts. While these features can be helpful indicators, they are not failproof. Honest users oftentimes travel and login from different devices at unusual times. Similarly, an attacker may attempt to mimic the login patterns of legitimate users. The online attacker can also submit guesses from a wide variety of IP addresses and geographical locations, e.g., using a botnet. 




\vspace{-0.1cm}


\subsection{Contributions} 


We introduce $\DALock$, a novel \underline{D}istribution-\underline{A}ware throttling mechanism that can achieve a better balance between usability and security. The key intuition behind $\DALock$ is to base lockout decisions on the {\em popularity} of the passwords that are being guessed. An online attacker will typically want to attempt the most popular passwords to maximize their chances of success. By contrast, when an honest user miss-types (or miss-remembers) their password, the attempt is less likely to be a globally popular password. In addition to keeping track of $\strikeThresholdOfU{u}$ (the number of consecutive incorrect login attempts), $\DALock$ keeps track of a ``hit count'' $\hitCountThresholdOfU{u}$ for each user $u$, where $\hitCountThresholdOfU{u}$ intuitively represents the cumulative probability mass of all incorrect login attempts for user $u$'s account. When $\hitCountThresholdOfU{u}$ exceeds the threshold $\hitCountThreshold$, we decide to lock the account. 



\paragraph{Example 1: Usability} \textbf{Figure}~\ref{figure:introduction_security} compares the usability of $\DALock$ with the standard 3-strikes mechanism. In this example scenario, our user John Smith registers an account with the somewhat complicated password ``J.S.UsesStr0ngpwd!'' based on the story ``John Smith uses a strong password.''. Later, when John tries to login into his account, John remembers the basic story, but not the exact password. Did he use his first name and his last name? With or without abbreviation? Did he add a punctuation mark at the end? Which letters are capitalized? If we use the 3-strikes mechanism, John Smith will be locked out quickly, e.g., after trying the incorrect password guesses ``JohnUseStrongPassword,'' ``JohnUsesStrongPassword,'' and ``JohnUsesStrongpwd.'' However, since none of these passwords is overly popular $\DALock$ would allow our user to continue attempting to login until he recovers the correct password. 



\paragraph{Example 2: Security} \textbf{Figure}~\ref{figure:introduction_security} compares $\DALock$ with the 10-strikes mechanism. In this scenario, our user registers an account with a weak password ``letmein.'' Because the password is globally popular, it is likely that an online attacker will attempt this password within the first $10$ guesses and break into the account. By contrast, $\DALock$ will quickly lock down the account after the attacker submits two globally popular passwords. 


To deploy $\DALock$, we need a \textit{frequency oracle} to estimate the frequency of each incorrect login attempt to update $\hitCountThresholdOfU{u}$. We propose two implementations: password strength models (e.g., $\ZXCVBN$~\cite{USENIX:Wheeler16}) and a differentially private count sketch data structure. Of course, no frequency oracle will perfectly estimate the true strength of a password and the attacker may try to exploit passwords that are over/underestimated by frequency oracle. We introduce the password knapsack problem to model the optimal (untargeted) attack against $\DALock$. Intuitively, the attacker will try to find a subset of passwords to check which maximizes his success rate subject to the constraint that the total estimated hit count does not exceed the threshold $\hitCountThresholdOfU{u}$.  While password knapsack is $\NP$-Hard, we show that a simple heuristic algorithm works well on empirical datasets. 


We then evaluate $\DALock$ empirically by simulating an authentication server in the presence of an online password attacker comparing $\DALock$ with the traditional $\strikeThreshold$-strikes mechanism for $\strikeThreshold \in \{3,10\}$. In our simulations, we use the password knapsack problem to model the behavior of the attacker and we model honest user login attempts/mistakes using a simple model based on prior empirical studies of password typos~\cite{CCS:CWPCR17,SP:CAAJR16}. Our experiments show that when the hit count threshold $\hitCountThreshold$ is tuned appropriately, $\DALock$ significantly outperforms $\strikeThreshold$-strikes mechanisms. In particular, when user accounts are under attack, we find that the fraction of accounts that are compromised is significantly lower for $\DALock$ than classic $\strikeThreshold$-strikes mechanisms --- even for the strict $\strikeThreshold$=3 strikes policy. We also evaluate the unwanted lockout rate of user accounts that are not under attack. We find that the unwanted lockout rate for $\DALock$ is much lower compared to $\strikeThreshold$=3 strikes mechanism. The unwanted lockout rate for $\DALock$ and the more lenient $\strikeThreshold$=10 strikes mechanism were comparable. We also evaluate the performance of $\DALock$ when the organization bans the top $B$ most popular passwords to encourage users to select stronger passwords. We find that $\DALock$ continues to outperform the traditional $\strikeThreshold$=3 strikes mechanism in terms of both usability and security. A more detailed description of our experiments can be found in \textbf{section}~\ref{section:experimentalresult}.







% Old stuff: However, $\DALock$ performs best when we instantiate with a differentially private count sketch. On a positive note we show that even if the differentially private count sketch is trained on a small subset of user passwords that the estimates will still be high enough for $\DALock$ to be effective.



% In our experimental evaluation we do not rely on features such as IP address or geographical location to detect attackers though we remark that $\DALock$ could be 



%\jeremiah{Move the explanation below elsewhere:	In our simulations honest user's periodically attempt to login to their accounts following a Poisson arrival process. Each honest user selects his password from an empirical password distribution e.g., based on data from various datasets to simulate login mistakes each honest user additionally selects four other passwords from the same distribution i.e., passwords for other accounts. Every time a user attempts to login in our simulation he will make a mistake with some fixed probability. These mistakes might include typing errors (e.g., capitalization, CAPSLOCK, keyboard proximity errors) or the user might forget which of his five passwords to use.}


%\wuwei{Moved to simulating users}



%To achieve better usability without compromising security, we propose a novel password \underline{D}istribution \underline{A}ware throttling mechanism: DALock. On the usability side, $\DALock$ tolerates users' honest mistakes by granting them more chances to enter their correct passwords as demonstrated in \textbf{Figure}~\ref{figure:introduction_usability}. Surprisingly on the security side, dictionary attacker does not benefit from $\DALock$ and still subject to limited chance of guessing (\textbf{Figure}~\ref{figure:introduction_security}). In this work, we theoretically proved that adversaries are challegned by a computational difficult task to perform optimal attacks. We further empirically measure the performance of $\DALock$ for both security and usability on three real datasets. 


\begin{figure}[htb]
	\begin{center}
		\includegraphics[height=1in,width=0.48\linewidth]{Figures/Introduction/Usability.pdf}\includegraphics[height=1in,width=0.48\linewidth]{Figures/Introduction/Security.pdf}		\caption{Security(L) \& Usabilty(R) Comparison}\label{figure:introduction_security}
	\end{center}
	\vspace{-0.8cm}
\end{figure}




%In this work, we focus on the online dictionary attack scene. We assume that dictionary attackers are trying to optimize their chance of success globally. i.e., the adversaries are trying to break as many accounts as possible. In this work, we assume they have reasonable efficient computational powers, for example, a bitcoin mining machine, to perform non-trivial task easily (problems not in NP). On top of that, the adversaries have precise knowledge of the distribution of users' passwords and deployed security parameters of $\DALock$ mechanism. 





% !TEX root = main.tex

% Differential Private Releasing
\vspace*{-\baselineskip}
\vspace*{-\baselineskip}
\section{Related Work and Background}\label{sec: relatedwork}

\vspace*{-\baselineskip}

\subsection{Authentication Throttling} \label{related: Throttling}

\vspace*{-\baselineskip}

%{\mypara{K-strikes Mechanism} K-strikes mechanism is a straight-forward implementation for authentication throttling. As its name suggests, throttling occurs when $\strikeThreshold$ consecutive incorrect login attempts are detected.  To reduce the cost of expensive overhead caused by unwanted throttling, Brostoff~\cite{brostoff2003ten} et al. suggest setting threshold $\strikeThreshold$ to be 10 instead of 3. They argue the increment risk is limited when a strong password policy is enforced. However, this argument is challenged by empirical analyses of password composition policies~\cite{KSKMBCCE:SIGCHI11, BKPS:ACMEC13}. Many password composition policies do not rule out all lw entropy password choices. For instance, it turns out that banning dictionary words does not increase entropy as expected~\cite{KSKMBCCE:SIGCHI11}.  } 

\mypara{Feature-Based Mechanism} Modern throttling mechanisms~\cite{sandhu2005system, gordon2014efficiently} often use features such as geographical location, IP-address, device information, etc., to detect unusual activities. These features can be used to train sophisticated machine learning models to help distinguish between malicious and benign login attempts~\cite{NDSS:FJDBG16}. $\DALock$ takes an orthogonal approach and relies instead on the popularity of the password guesses instead of potentially confidential user profiles. One can combine those models with a rigorous throttling system for better performance.

\mypara{Password Distribution-Aware Throttling} In an independent line of work, Tian et al.~\cite{EuroSP:THS19} developed an IP-based throttling mechanism {called StopGuessing} that exploits differences between the distribution of honest login attempts and malicious guesses. In particular, they propose to ``silently block'' login attempts from a particular IP address $ip$ if the system detects too many popular passwords being submitted from $ip$. In more detail, StopGuessing uses a data structure called the binomial ladder filter~\cite{SchHer:MSR18} to (approximately) track the frequency $F(pw)$ of each incorrect password guess $pw$. For each IP address $ip$, the StopGuessing protocol maintains an associated counter $I_{ip} = \displaystyle{\sum_{pw \in \mathcal{P}} F(pw)}$ where $\mathcal{P}$ is a list of incorrect password guesses that have been (recently) submitted from $ip$ --- $I_{ip} $ can be updated without storing $\mathcal{P}$ explicitly. Intuitively (and oversimplifying a bit) if $I_{ip}$ exceeds a predefined threshold $T$, then login attempts from address $ip$ are silently blocked, i.e., even if the attacker (or honest user) submits a correct password, the system will respond that authentication fails. The authors also suggest protecting accounts with weak passwords by setting a user-specific threshold $T(F(pw_{u}))$ based on the strength $F(pw_{u})$ of the password $\PasswordOfU{u}$ of user $u$. Now, if $I_{ip} > T(F(pw_{u}))$, the system will silently reject any password from address $ip$. Both StopGuessing and $\DALock$ exploit differences between the distribution of user passwords and attacker guesses. One of the key differences is that StopGuessing focuses on identifying malicious IP addresses (by maintaining a score $I_{ip}$ for each IP address $ip$) while $\DALock$ focuses on protecting individual accounts by maintaining a ``hit-count'' parameter $\hitCountThreshold{u}$ for each user u. There are several other key differences between the two approaches as well. First, in $\DALock$, the goal of our frequency oracle (e.g., count sketch, password strength meter) is to estimate the \textit{total fraction} of users who have actually selected that particular password --- as opposed to estimating the frequency with which that password has been {\em recently} submitted as an incorrect guess. Second, $\DALock$ does not require silent blocking of login attempts, which could create usability concerns if an honest user is silently blocked when they enter the correct password.  
\vspace*{-\baselineskip}
\vspace*{-\baselineskip}
\subsection{Passwords} \label{related: Passwords}  
\vspace*{-\baselineskip}
\mypara{Password Distribution} To analyze online statistical guessing attacks it is important {to understand} the distribution of user passwords. Password distributions have been extensively studied since the last decades~\cite{SP:Bonneau12,FloHer:WWW07,DavKev:WWW12}. Wang et al.~\cite{EPRINT:WJHW14,TIFS17:WCWPXG,ESORICS:WanWan16} observed that Zipf's law distributions nicely fit leaked password corpora  and Blocki et al.~\cite{SP:BloHarZho18} later found that the same for the differentially private Yahoo! password frequency corpus~\cite{SP:Bonneau12,NDSS:BloDatBon16}. Other work has used password cracking models and/or statistical techniques to characterize the password distribution. 

\mypara{Password Typos}  Recent studies~\cite{CCS:CWPCR17,SP:CAAJR16} from Chatterjee et al. have summarized probabilities of making (various types of) typos when one enters his or her password based on users' studies. Based on the empirically measured data, they proposed two typo-tolerant authentication {mechanisms and demonstrate that typo-correction does not come at the cost of security ---} similar mechanisms have already been deployed in the industry~\cite{News:AmazonTypo}.     {In our usability simulations for $\DALock$ we leverage the findings of \cite{CCS:CWPCR17,SP:CAAJR16}  to help simulate honest user mistakes during authentication. }


\vspace*{-\baselineskip}
\vspace*{-\baselineskip}
\subsection{Eliminating Dictionary Attacks} 
\vspace*{-\baselineskip}
\mypara{Increasing Cost of Authentication} Pinkas and Sanders~\cite{CCS:PinSan02} proposed using puzzles (e.g., CAPTCHAs) as a way to stop online password crackers. CAPTCHAs are hard AI challenges meant to distinguish people from bots~\cite{EC:vBHL03}. For example, reCAPTCHA~\cite{von2008recaptcha} has been widely deployed, e.g., Google, Facebook, Twitter, CNN, etc. If we assume that CAPTCHAs are only solvable by people, it is possible to mitigate automated online attacks without freezing users' accounts~\cite{SP:BBFNJ10,CCS:BurMarMit11}. Nevertheless, an attacker can always pay humans to solve CAPTCHA challenges~\cite{captchaSolver}. Besides, sophisticated CAPTCHA solvers~\cite{CCS:YTFZFX18} powered by neural networks make it increasingly challenging to design CAPTCHA puzzles that are also easy for a human to solve.  Golla et al.~\cite{SOUPS:GBD17} proposed a fee-based password verification system where a small deposit is necessary to authenticate, which is refunded after successful authentication. A password cracker risks losing its deposit if it is not able to guess the real password.  

\mypara{Eliminating Popular Passwords} One mediation for dictionary attacks is eliminating the existence of weak or popular passwords. Password composition policy is a common approach, but efforts to force users to pick strong passwords by requiring users to include numbers, capital letters, and/or special symbols have shown limited success ~\cite{KSKMBCCE:SIGCHI11, BKPS:ACMEC13}. An alternate approach of Schechter et al.~\cite{HTS:SchHerMit10} is to ban passwords if and only if too many users have picked them using a count-sketch data structure for frequency estimation. A theoretical model by Blocki et al. \cite{BKPS:ACMEC13} shows that this is the optimal approach to boost the minimum entropy of the password distribution.  


\vspace*{-\baselineskip}




% !TEX root = main.tex


\vspace*{-\baselineskip}
\section{Preliminaries} \label{section:Prelinmaries}

\vspace*{-\baselineskip}
% Differential Privacy

\subsection{Count Sketch}\label{section:Prelinmaries-CountSketch} %Done
\vspace*{-\baselineskip}
\wuwei{Reviewer A mentioned definition of w is unclear? Shall we state that w is the width of the Count Sketch?}
The count sketch~\cite{ICALP:ChaCheFar02} is a succinct data structure which allows for one to quickly obtain an approximation of the frequency of any item in a dataset.   The state $\sigma: \mathsf{R}^{d\times w} \times \mathsf{R}$ of a count sketch ($\CountSketch$) is represented by a two-dimensional $d\times w$ array $\CountSketchArray$ and a total frequency counter $\CountSketchCounter$. The data strucutre additionally uses $d+1$ hash functions ($h_1, \ldots, h_d$, $h_{\pm}$) with the first $d$ functions chosen uniformly at random from a pairwise-independent family:  
	%\begin{center}
		$$h_1, \cdots, h_d : \{pw\} \rightarrow \{1\cdots w\}~ ,~~ \&~~ h_{\pm} : \{pw\} \rightarrow \{1, -1\}$$
	%\end{center}
	
In this work, we consider the following four classic count sketch APIs: Initialize, Add, Estimate, and TotalFreq. 


\mypara{$\sigma_{0} \leftarrow$ Initialize($d,w$)}: This API initializes and returns a count sketch of state $0^{d\times w}\times 0$, i.e., an all-zero table.


\mypara{$\sigma_{new} \leftarrow $ Add($pw, \sigma$):} Intuitively, the Add operation returns an updated count sketch state $\sigma_{new}$ in which the frequency count of password $pw$ increases by 1.

Given a multiset $\SampledData{\AllUser} = \{pw_1,\cdots,pwd_N\}$, we use the following notation $\sigma_{\SampledData{\AllUser}} =  Add(\SampledData{\AllUser},\sigma) \\= Add(pw_1, Add(pw_2,Add(pw_3,\cdots))$ to ease presentation. When the context is clear we also omit the subscript $\SampledData{\AllUser}$ and simply use $\sigma$ to denote $\sigma_{\SampledData{\AllUser}}$.


\mypara{Estimate($pw,\sigma$)}: This interface returns the estimated frequency of a password $pw$ based on the given count sketch state $\sigma$.


To implement $\DALock$ with high accuracy, we want the estimator to have the following correctness property: $\EstimateF{pw}{\sigma} \approx \TrueFInD{pw}{\SampledData{\AllUser} }$, where $\TrueFInD{pw}{\SampledData{\AllUser}}$ denotes the actual frequency of $pw$ in $\SampledData{\AllUser}$.


\mypara{TotalFreq($\sigma$)}: This operation returns the total number of passwords based on state $\sigma$.


Based on the above definition, we denote the \emph{estimated popularity} of a password $pw$ by $\sigma$ with $\EstimateP{pw}{\sigma} = \frac{\EstimateF{pw}{\sigma}}{\text{TotalFreq}(\sigma)}$. For the rest of the discussion, we sometimes omit $\sigma$ when there is no ambiguity to simplify the presentation. e.g. $\EstP{pw}$ = $\EstimateP{pw}{\sigma}$. In addition, we allow the above APIs to take a set of passwords as an argument and return the summed results. i.e. $\EstP{S} = \displaystyle{\sum_{pw \in S} \EstP{S}}$. 




\vspace*{-\baselineskip}

%widely used in the tasks for finding frequent items such as popular passwords~\cite{CCS:NaoPinRon19}, homepage settings~\cite{CCS:ErlPihKor14}. . 
\subsection{Differential Privacy} \label{section:Prelinmaries-DiffernetialPrivacy}
\vspace*{-\baselineskip}
While the succinct count-sketch data structure is a useful tool to approximate the freqeuncy of a particular password in the dataset, its usage raises a natural privacy concern. Could the attacker infer anything about a particular user's password from the count-sketch $\sigma$ if the authentication server was breached? We address these concerns by using a differentially private count sketch. Differential privacy~\cite{ECS:Dwork11} is a compelling mathematical definition of privacy that has begun to see industrial deployment\cite{CCS:ErlPihKor14}. It is often viewed as a gold standard for data privacy.  In this work, we adopt differentially private count sketches to reduce the risk of privacy leakage. Based on our notion of count sketch, one can define differential privacy as follows.

\begin{definition}[{$\epsilon$-Differential Privacy~\cite{ECS:Dwork11}}] \label{def:diff}

	A randomized mechanism $\mathcal{M}$ gives $\epsilon$-differential privacy if for any pair of neighboring datasets $\SampledData{\AllUser}$ and $\SampledData{\AllUser}'$, and any $\sigma \in \mathit{Range}(\mathcal{M})$,
	$$\Pr{\mathcal{M}(\SampledData{\AllUser})=\sigma} \leq e^{\epsilon}\cdot \Pr{\mathcal{M}(\SampledData{\AllUser}')=\sigma}.$$
\end{definition}


We consider two datasets $\SampledData{\AllUser}$ and $\SampledData{\AllUser}'$ to be neighbors i.f.f. either $\SampledData{\AllUser}=\SampledData{\AllUser}' + pw_u$ or $\SampledData{\AllUser}'=\SampledData{\AllUser} + pw_u$, where $\SampledData{\AllUser} +pw_u$ denotes the dataset resulted from adding the tuple $pw_u$(a new password) to the dataset $\SampledData{\AllUser}$. We use $\SampledData{\AllUser}\simeq \SampledData{\AllUser}'$ to denote two neighboring datasets. This protects the privacy of any single tuple(password) because adding or removing any single password results in $e^{\epsilon}$-multiplicative-bounded changes in the probability distribution of the output. If an adversary can make a certain inference about a password based on the output, then the same inference is also likely to occur even if the password does not appear in the dataset.

\mypara{Laplace Mechanism} 
The Laplace mechanism is a classic tool to achieve differential privacy. It computes a differentially private state $\sigma$ based on dataset $\SampledData{\AllUser}$ by adding random Laplace noise. The magnitude of the noise depends on $\mathsf{GS}_\sigma$, the \emph{global sensitivity} or the $L_1$ sensitivity of $\sigma$.  $\mathsf{GS}_\sigma$ quantifies the maximum impact on $\sigma$ if one adds or removes any record. 


\mypara{Differentially Private Count Sketch} Given a $\CountSketch$ state $\sigma$, adding (removing) any password $pw$ to(from) $\sigma$ can result in at most d + 1 changes for $l_1$ norm. Because each $pw$ contributes to d entries in the $d \times w$ table $\CountSketchArray$ and total count $\CountSketchCounter$. Therefore, \replaced{to}{To} release $\sigma$ with privacy budget $\epsilon$, it suffices to add $\Lap(\frac{d+1}{\epsilon})$ to all entries in $\sigma$. This noise can be added during initialization. Formally, we use \mypara{$\sigma_{dp}$ $\leftarrow$ DP($\epsilon, \sigma$)} to denote a function which adds laplace noise $\Lap(\frac{d+1}{\epsilon})$  to the count sketch state $\sigma$ to obtain $\epsilon$-differentially private count sketch state $\sigma_{dp}$. This step can be carried out immediately after initialization. 

\mypara{Differential Privacy in Passwords} Naor et al.\cite{CCS:NaoPinRon19} designed a locally differentially private mechanism to identify the most popular passwords in a distribution. Blocki et al.~\cite{NDSS:BloDatBon16} developed a differentially private mechanism for integer partitions and used this to release a private summary of the Yahoo! password dataset. StopGuessing\cite{EuroSP:THS19} uses a binomial ladder to identify ``heavy hitters'' (popular passwords), though the data-structure does not provide any formal privacy guarantees such as differential privacy and can overestimate the frequency of recently popular passwords. The binomial ladder is not suitable for $\DALock$ as it provides a binary classification, i.e., either the password is a ``heavy hitter'' or it is not. For $\DALock$ requires a more fine-grained estimate of a password’s popularity. We elect to use the count (median) sketch~\cite{ICALP:ChaCheFar02} data structure in this work as it is invariant to the order in which passwords are added, and because it can easily be modified to preserve differential privacy. 

\vspace*{-\baselineskip}
\subsection{Notation Summary}
\vspace*{-\baselineskip}
In this section, we summarize frequently used notations in this paper across all sections in \textbf{Table}~\ref{table: notation}, Appendix.  For a password $pw \in \AllPassword$, we use $\TrueP{pw}$ to denote the probability each user selects the password $pw$. We assume that there is some underlying distribution over user passwords and use $\TrueP{pw}$ to denote the probability of the password $pw \in \AllPassword$. It will be convenient to assume that all passwords $ \AllPassword = \{pw_1,pw_2,\ldots \}$ are sorted in descending order of probability, i.e., so that $\TrueP{pw_1} \geq \TrueP{pw_2} \ldots $. %We will use $pw_r = \TrueP{pw_r}$ to denote the probability of the $r$th most likely password in the distribution. 


We use $\AllUser = \{u_1,\ldots,u_N\}$ to denote a set of $N$ users and $\SampledData{\AllUser} \subseteq \AllPassword$ is a multiset of user passwords, i.e., $\SampledData{\AllUser} = \{pw_{u_1},\ldots,pw_{u_N}\}$. We typically view $\SampledData{\AllUser}$ as $N$ independent samples from an underlying distribution over $\AllPassword$ and write $\TrueF{pw, \SampledData{\AllUser} }= \left| \left\{ i ~:~pw_{u_i} = pw \right\} \right|$ to denote the number of times the password $pw$ was observed in our sample. We often omit $\SampledData{\AllUser}$ in the notation when the dataset is clear from the context and simply write $\TrueF{pw}$. 

We remark that $\TrueP{pw} = \frac{\mathbb{E}\left[ \TrueFInD{pw}{ \SampledData{\AllUser}}\right]}{N}$ and thus for popular passwords we expect that the estimate $\TrueP{pw} \approx  \frac{\TrueF{pw, \SampledData{\AllUser}}}{N}$ will be accurate for sufficiently large $N$. However, because the underlying password distribution is unknown and an authentication server cannot store a plaintext encoding of $\SampledData{\AllUser}$ we will often use other techniques to estimate  $\TrueP{pw}$ and/or $\TrueF{pw, \SampledData{\AllUser}}$. In particular, we consider a count sketch data structure $\CountSketch$ trained on $\SampledData{\AllUser}$ (or a small subsample of $\SampledData{\AllUser}$), which allows us to generate an estimate $\EstP{pw}$ for the popularity of each password. Similarly, we can also use password strength meters to compute $\EstP{pw}$ to estimate $\TrueP{pw}$.


%========= Above In Appendix ===========



% !TEX root = main.tex

\section{The $\DALock$ Mechanism}\label{sec:DALockAlgorithm} %Done
In this section, we present the $\DALock$ mechanism, discuss how $\DALock$ might be implemented and the strategies that an attacker might use when $\DALock$ is deployed. Intuitively, $\DALock$ punishes incorrect password guesses more harshly if the guessed password $pw$ is overly popular since an attacker will want to submit popular password guesses to maximize their chances of cracking the users' passwords.

\subsection{$\DALock$} %Done
In the classic $K$-strike throttling mechanism we keep track of a parameter $\KOfU$ which tracks the number of consecutive incorrect login attempts for each user $u$. After each consecutive login attempt the parameter is $\KOfU$ and the parameter $\KOfU$ is reset to $0$ whenever $u$ authenticates successfully. If we ever have $\KOfU \geq \strikeThreshold$ then throttling mechanism kicks in and the authentication server will lock down the account until the user takes some action\footnote{For example, the user might be asked to  resetting his password via e-mail or wait for some fixed amount of time. In some settings the user might simply be asked to solve a CAPTCHA challenge. The latter approach has some usability advantages and security drawbacks e.g., a malicious password might pay to solve the CAPTCHA challenges so that he can continue attempting to guess the user's password}.

The key-idea behind $\DALock$ is to additionally maintain an extra ``hit count'' variable $\hitCountThresholdOfU{u}$ for each user $u$. Intuitively, $\hitCountThresholdOfU{u}$ measures the total probability mass of all incorrect password guesses submitted for user $u$. Initially, when a new user registers we will have $\hitCountThresholdOfU{u}=0$ (and $\KOfU =0$). After each attempted login with an incorrect password $pw \neq pw_u$ the hit count is incremented so that $\hitCountThresholdOfU{u}+= \EstP(pw)$. Here, $\EstP(pw)$ denotes an estimate for the probability of the password $pw$ so that incorrect passwords are punished more severely when $pw$ is an overly popular password. Unlike the consecutive strike parameter $\KOfU$ which is reset to $0$ after each successful login, the hit count parameter can only be incremented. $\DALock$ throttles $u$'s account if the ``hit count'' exceeds $\hitCountThreshold$ (i.e., $\hitCountThresholdOfU{u} \geq \hitCountThreshold$) or if there are too many consecutive mistakes (i.e., $\strikeThresholdOfU{u} \geq \strikeThreshold$)  For example, suppose that the (estimated) probability of the passwords ``aaa," ``bbb" and ``ccc'' were 3\%, 1.7\% and 0.8\%. If a user registers with a password ``ddd'' and then attempts to login with the previous three passwords, then $\hitCountThresholdOfU{u}$ will be set to $0.055=0.03+0.017+0.008$. 

Each time the user (or attacker) attempts to login with a password $pw$ the response will either be (1) ``locked'' if $\hitCountThresholdOfU{u} \geq \hitCountThreshold$ or if $\strikeThresholdOfU{u} \geq \strikeThreshold$, (2) ``correct'' if the guessed password matches the user password i.e., $pw = pw_u$\footnote{To ease presentation we omit the description of the password hashing algorithm when we describe the authentication server. In practice, we recommend that the authentication server only stores salted password hashes using a moderately expensive key derivation function to increase guessing costs for an offline attacker.} or (3) ``incorrect password'' otherwise. We demonstrate the login flow in \textbf{Algorithm}~\ref{algorithm:DALock}, Appendix. We remark that the authentication server could intentionally blur this distinction between cases (1) and (3), but that this comes at a usability cost e.g., an honest user would be annoyed if they were repeatedly informed that their password is incorrect whenever the account is actually locked.



%We start the discussion by first introducing the classic $K$-strike throttling mechanism. As its name suggests, throttling happens when one makes $K$ consecutive mistakes. To achieve this goal, $K$-strike mechanism maintains a counter $\KOfU$ for each user u to record how many incorrect attempts were made. Typically, $\KOfU$ is reset to 0 whenever u logins successfully. Some systems also reset $\KOfU$ to 0 after a certain amount of time and/or maintain separate counters for different IP addresses. In our experiments we consider the $K$-strike mechanism where $\KOfU$ is reset to 0 only after successful login attempts. 

%The $\DALock$ mechanism maintains an extra ``hit count" variable $\hitCountThresholdOfU{u}$ for each user. Intuitively, $\hitCountThresholdOfU{u}$ measures the total probability mass of all incorrect password guesses submitted for user $u$. For example, suppose that the (estimated) probability of the passwords ``aaa," ``bbb" and ``ccc'' were 3\%, 1.7\% and 0.8\%. $\hitCountThresholdOfU{u}$ is set to $0.055=0.03+0.017+0.008$ if one fails to login into the account with those passwords. We demonstrate the login flow in \textbf{Algorithm}~\ref{algorithm:DALock}, Appendix. Here, $\PasswordOfU{u}$ denotes the actual password of user $u$ and $\hitCountThreshold$ (resp. $\strikeThreshold$) are global threshold parameters. $\DALock$ locks $u$'s account if the ``hit count'' exceeds $\hitCountThreshold$ (i.e., $\hitCountThresholdOfU{u} \geq \hitCountThreshold$) or if there are too many consecutive mistakes (i.e., $\strikeThresholdOfU{u} \geq \strikeThreshold$). \footnote{We omit the description of the password hashing algorithm to ease presentation. Any secure implementation would need to use a salted password hash algorithm with plenty of key-stretching to increase guessing costs for an offline attacker.}


{\noindent \bf Remark:} One could optionally consider initializing the hit count parameter $\hitCountThresholdOfU{u}$ based on the strength of the user's password. For example, if $u$ registers with a weak password then we might initialize $\hitCountThresholdOfU{u} = \hitCountThreshold/2$ for stronger protection i.e., so that the account is locked down faster. Similarly, a user with a strong password might be awarded by setting $\hitCountThresholdOfU{u} = \hitCountThreshold$. However, because $\hitCountThresholdOfU{u}$ and $\strikeThresholdOfU{u}$ are stored on the authentication server this would leak information about the strength of $pw_u$ to an offline attacker e.g., if an offline attacker sees that $\hitCountThresholdOfU{u} = \hitCountThreshold/2$ he might reasonably infer that the user picked a weak password. \footnote{One could potentially avoid storing $\hitCountThresholdOfU{u}$ unencrypted if one is willing to implement a silent lockout policy where the user cannot distinguish between an incorrect guess and a locked account, but we wish to avoid solutions that blur this distinction.}  



\subsection{$\DALock$ Authentication Server} %Done
To implement $\DALock$ we need an efficient way to estimate the probability $\EstP(pw)$ of each incorrect password $pw$. We consider several instantiations of this frequency oracle. One option is to use password strength meters such as $\ZXCVBN$ or more sophisticated password cracking models e.g., Markov Models, Probabilistic Context Free Grammars, or Neural Networks. Another naive approach would be to simply maintain a plaintext list of all user passwords along with their frequencies. However, this approach is inadvisable due to the risk of leaking this plaintext list. Herley and Schechter~\cite{HTS:SchHerMit10} proposed the use of the Count-Sketch data-structure which would allow us to estimate the frequency of each password without explicitly storing a plaintext list although there are no formal privacy guarantees to this approach. We chose to adopt a Differentially Private Count-Median-Sketch. The authentication server initializes the Count-Sketch $\sigma_{dp}$ $\leftarrow$  DP($\epsilon, \sigma$) by adding Laplace Noise to preserve $\epsilon$-differential privacy and each time a new user $u$ registers a new password  $pw_u$ would be added to the Count Sketch.

%There are several ways to accomplish this task. Password strength meters such as $\ZXCVBN$ can provide reasonably accurate measurement of password strength\cite{CCS:GolDur18}; nevertheless it can be a usability concerns for a user if they accidentally enters a typo result in low entropy. Another naive approach would be to simply maintain a plaintext list of all user passwords along with their frequencies. However, this approach is inadvisable due to the risk of leaking this plaintext list. Herley and Schechter~\cite{HTS:SchHerMit10} proposed the use of the Count-Sketch data-structure which would allow us to estimate the frequency of each password without explicitly storing a plaintext list. However, there are no formal privacy guarantees. We chose to adopt a Differentially Private Count-Median-Sketch. In our experiments we find that the Laplace Noise added to preserve differential privacy does not adversely affect the performance of $\DALock$, and we also found that Differentially Private Count-Median-Sketch outperforms its differentially private counterparts Count-Mean and Count-Min. In the rest of this, we simply refer to Count-Median-Sketch as Count Sketch.

We remark that maintaining a Differentially Private Count-Sketch has many other potentially beneficial applications e.g., one could use the Count-Sketch to ban weak passwords~\cite{HTS:SchHerMit10} and/or to help identify IP addresses associated with malicious online attacks~\cite{EuroSP:THS19}. One disadvantage is that the attacker will also be able to view the Count-Sketch data-structure if the data-structure is leaked. The usage of differential privacy helps to minimize these risks. Intuitively, differential privacy hides the influence of any individual password ensuring that an attacker will not be able to use the Count-Sketch data-structure to help identify any unique passwords. However, an attacker may still be able to use the data-structure to learn that a particular password is globally popular (without linking that password to a particular user). We argue that this is not a major risk as most attackers will already know about globally popular passwords e.g., from prior breaches. 

% an efficient data structure to accurately, privately, and securely estimate the probability of each user password. We adopt Differential Private Count-Median-Sketch to store password distribution as it meets all the expectations. Based on our experiments, Count-Median-Sketch has the lowest $l_1$ error on average compare to Count-Mean-Sketch and Count-Min-Sketch. 

%

%There are multiple ways to estimate password frequencies and popularities besides Count-Sketch. For instance, one can use existing password corpora, and password strength meter to estimate the distribution of the password; however, one advantage of using Count-Sketch is it can be used in other applications,  e.g., to help ban weak passwords or to identify malicious IP addresses\cite{EuroSP:THS19}.  A disadvantage is the potential risks associated with leaking the Count-Sketch data-structure if there is a breach. The usage of DP helps to minimize (but may not completely eliminate) these risks. 

%{\bf Using/Releasing Differentially Private Count Sketch: } 
%In this work, we consider the actual usage of a distribution in $\DALock$ as releasing it. We assume that $\Adversary$ can potentially acquire knowledge based on the feedback of $\DALock$ and therefore it is crucial to ensure privacy protection. In this work we consider the centralized model of differential privacy. i.e. We assume the server periodically updates the a securely stored and release a differentially private version of it by batching. One natural question is whether such approach is sustainable for essential multiple releases in long term. Our results show that one can achieve high accuracy with low privacy budget e.g. $\epsilon = 0.1$ which reduce the cost of cumulative privacy loss.














% !TEX root = main.tex
\vspace*{-\baselineskip}
\section{Experimental Design} %Done

We evaluate the performance of $\DALock$ through an extensive battery of empirical simulations. In this section, we describe the modeling choices we made when designing our experiments. To simulate the authentication ecosystem, we need to simulate honest users' behavior, the authentication server running $\DALock$, and an online attacker. 

Briefly, when simulating users, we need to model the distribution over users’ passwords, the distribution over honest login mistakes (e.g., typos or recall errors), and the user's login schedule. When simulating the distribution over users’ passwords, we use multiple empirical datasets to define the underlying password distribution. We use a Poisson arrival process to model the frequency of user login attempts~\cite{AC:BloBluDat13}. Our model for users’ mistakes is informed by recent empirical studies of password typos~\cite{CCS:CWPCR17,SP:CAAJR16} and is augmented to simulate other mistakes, i.e., recall errors.  The key question for simulating an authentication server running $\DALock$ is how the (password) frequency oracle $\EstP{\cdot}$ is implemented. We consider two concrete implementations: password strength models~\cite{ USENIX:Wheeler16,USENIX:USBCCKKMMS15,USENIX:MUSKBCC16} (e.g., $\ZX$, Markov Models, Neural Networks) and (differentially private) count sketches. When simulating the attacker, we consider an untargeted one who knows the distribution over user passwords as well as the $\DALock$ mechanism --- including the frequency oracle $\EstP{\cdot}$. We leave the question of tuning DALock to protect against targeted online attackers~\cite{CCS:WZWYH16} as an important direction for future research. We elaborate on each of these key model components below.  We begin \deleted{by} with an overview of the empirical datasets $\SampledData{\AllUser}$ that we used in our experiments.

\vspace*{-\baselineskip}
\subsection{Experimental Datasets}\label{section:experiment:experiment_dataset} 
\vspace*{-\baselineskip}
In this work, we use multiple real-world password datasets. See Table~\ref{table: datasetsummary} for a summary of each dataset including (1) the total number of unique passwords in the dataset, (2) the total number of user accounts in the dataset, (3) the probability of the most popular password, and (4) the cumulative probability of the top 10 passwords. Except for the differentially private Yahoo! frequency corpus\footnote{Anonymized password histograms representing almost 70 million Yahoo! users who logged into their account during a $48$-hour window in May 2011 \cite{SP:Bonneau12}} , each dataset is the result of a data breach which was collected~\cite{SP:Bonneau12} and publicly released~\cite{NDSS:BloDatBon16} with permission from Yahoo!. We remark that this frequency corpus \textit{does not contain any plaintext passwords}, so we did not use password strength models in our experiments involving the Yahoo! dataset. %The frequency corpus consists of anonymized password histograms representing almost 70 million Yahoo! users who logged into their account during a $48$-hour window in May 2011.
 
%In \lazyref{Section}{section:experimentalresult}, we present the result using all datasets except LinkedIn


%\footnote{The LinkedIn dataset we used is a plaintext password corpus (\textit{partially}) recovered constructed from a leak in 2012. It contains approximately $68$ million cracked passwords, but the actual size of the leak is larger. Furthermore, there is a larger (differentially private) frequency corpus (without plaintext) based on $174+$ million passwords~\cite{harsha2020bicycle} that is publicly available. However, this dataset does not include any plaintext passwords. We chose to use the smaller dataset in our experiments so that we could evaluate with frequency oracles based on password models (e.g., $\ZXCVBN$, PCFGs, Neural Networks).}  and Yahoo. Results on these two datasets can be found in \lazyref{Appendix}{appendix:experimentalResults}.

Each dataset defines an empirical password distribution. In each of our experiments, we assume that this distribution matches the real (unknown) user password distribution from which these datasets were sampled. While the empirical distribution may not precisely match the real one, we stress that our analysis focuses on the most popular passwords in the distribution --- the ones that an attacker will try to guess. Because the datasets are all quite large ( the smallest dataset has over $0.5$ million passwords), standard concentration bounds imply that the true probability of a popular password in the distribution will almost certainly closely match the empirical probability.
\vspace*{-\baselineskip}

\begin{table}[h]
	
	\scalebox{0.90}{
		
		\begin{tabular}{|c|c|c|c|c|}
					
			\hline
					
			Dataset     & Passwords & Accounts & $\TrueP{pw_1}$ & $\TrueP{pw_{1-10}}$ \\ \hline
			
			
			Yahoo    & 33,895,873                     &  69,301,337            & 1.1\%  & 1.9\%                \\ \hline
			
			RockYou  & 14,341,564                & 32,603,388                      & 0.89\%  &2.1\%                   \\ \hline
			
			000webhost  & 10,587,915               & 14,960,642                      &  0.081\%&0.48\% \\ \hline
			
			LinkedIn&  6,840,885              & 68,361,064                    &1.53\% &2.82\%                        \\ \hline
			CSDN  & 4,037,268               & 5,908,494                       & 1.29\%     &3.72\%               \\ \hline
			
			clixsense  & 1,628,297               & 2,195,900 &  0.15\% & 0.7\% \\ \hline
			
			brazzers & 587,934 & 925,614 &0.58\% &1.13\%\\ \hline
			
			bfield  & 416,034& 539,434&  0.48\% & 1.97\% \\ \hline
					
		\end{tabular}	
		
	}
	
	\caption{Summary of dataset}\label{table: datasetsummary}
	
\end{table}

\vspace*{-\baselineskip}


\mypara{Ethics:} The datasets we used contain passwords that were previously stolen and subsequently leaked online. The use of such data raises critical ethical considerations; however, such password lists are already publicly available online, so our use of the data does not exacerbate the prior harm to users. We did not crack any new user passwords. Furthermore, the datasets we use have been cleaned of all identifying information beyond the passwords themselves.  In summary, we believe that our use of the leaked data will not exacerbate prior harm to users, and the lockout mechanism we develop and evaluate may help to protect user passwords in the future.

% Comment Unwanted text for later S&p Submission

% !TEX root = main.tex
\vspace*{-\baselineskip}
\subsection{Modeling Users} \label{section:ExperimentDesign-subsection:SimulateUser}

Our model to simulate honest users' behavior consists of three key components: user password selection, login frequency, and mistake model. 

\vspace*{-\baselineskip}
\subsubsection{Simulating Users’ Password Choices}\label{section:ExperimentDesign-subsection:SimulateUser-subsubsection:SimulatePasswordChoice}
\vspace*{-\baselineskip}
In each simulation, we fix a dataset that is used to simulate user password selection. In particular, a dataset consists of a multiset $\SampledData{\AllUser} = \{pw_1,\cdots,pw_N\}$ of $N$ passwords which can be compressed into pairs $(pw,  \TrueFInD{pw}{\SampledData{\AllUser} })$ where $\TrueFInD{pw}{\SampledData{\AllUser} }$ denotes the number of times the password $pw$ occurs in the dataset $\SampledData{\AllUser}$. Each dataset $\SampledData{\AllUser} $ induces an empirical distribution over users’ passwords where the probability of sampling each password $pw$ is simply $\frac{\TrueFInD{pw}{\SampledData{\AllUser}}}{N}$. 

\mypara{Simulating Password Choices} Each simulated user $u$ in our experiment samples 6 different passwords $pw_{u}^0,\ldots, pw_u^6$ \added{from} the empirical distribution and registers with the first sampled password $pw_u^{0}$. The remaining five passwords $pw_{u}^1,\ldots, pw_u^5$  intuitively represent the user's password for other websites and will be used to simulate recall errors (see \lazyref{Section}{section:ExperimentDesign-subsection:SimulateUser-subsubsection:SimulateUserMistake}).  



Move to appendix We remark that the Yahoo! dataset~\cite{SP:Bonneau12,NDSS:BloDatBon16} only contains frequencies without actual passwords i.e., instead of recording the pair $(pw,  \TrueFInD{pw}{\SampledData{\AllUser} })$ the dataset simply records $\TrueFInD{pw}{\SampledData{\AllUser} }$ . We generate a complete password dataset by designating a unique string for each password. As we avoid using password models like $\ZX$ to analyze $\DALock$ with the Yahoo! dataset since frequency estimation requires access to the original passwords. However, we are still able to evaluate $\DALock$ with the Yahoo! dataset using the Count-Sketch frequency oracle. 

\mypara{Ban-list} We additionally consider the setting where the authentication server chooses to ban users from selecting the top $B$ passwords, e.g., top 10 passwords. We use the normalized probabilities model~\cite{BKPS:ACMEC13} to simulate users' password selections under this restriction. In particular, we use rejection sampling to avoid sampling one of the top $B$ passwords. Equivalently, we can let $\SampledData{\AllUser, B}$ denote the dataset $\SampledData{\AllUser}$ with the $B$ most common passwords removed and sample from the empirical distribution corresponding to the updated dataset $\SampledData{\AllUser, B}$.







%Move to appendix \wuwei{This paragraph needs polishing}

%\mypara{Simulating Password Choices on Yahoo!} Simulating user's choice of password on Yahoo dataset involves an extra step: generate plaintext password string because it only contains the statistics of passwords. Notice that based on \textbf{Table}~\ref{table: datasetsummary},

%Yahoo contains more unique passwords than the rest two; therefore, it is impossible to map the dataset fully. To conquer this issue, we map password strings from RockYou to Yahoo as follows. Firstly, we sample users' choice of passwords from Yahoo based on its current distribution. Secondly, we map the top 20,000 passwords from RockYou to the top 20,000 passwords of Yahoo. For example, $pw_1$ and $pw_2$ from Yahoo are represented by ``123456" and ``12345" respectively, the top 2 passwords from RockYou. The goal of this step is to ensure an adequate string representation of popular passwords. Literatures\cite{EPRINT:WJHW14,TIFS17:WCWPXG,ESORICS:WanWan16,SP:BloHarZho18} suggests that Yahoo and RockYou both follow Zipf's law. Thus there are huge gaps among the top ranks. Thirdly, we map the rest of (RockYou) passwords to Yahoo according to their rankings and the users' choice.  For selected passwords, we map them from passwords with similar ranks in RockYou. On the other hand, we map the unselected passwords uniformly, roughly every 4, from RockYou. 
\vspace*{-\baselineskip}
\vspace*{-\baselineskip}


\subsubsection{Simulating User's Login Patterns}\label{section:ExperimentDesign-subsection:SimulateUser-subsubsection:SimulateLoginPattern} %Done
\vspace*{-\baselineskip}
To simulate users, we need to model the frequency with which our honest user attempts to login to the authentication server. In particular, we aim to simulate the login behaviors over a 180-day time span. For each user $u$, we want to generate a time sequence $0 < t_1^u < t_2^u < \cdots < 4320 = 180\times24$ where each $t_i^u \in \mathbb{N}$ represents the time (in hours) of the $i$th user visit. Following prior works (e.g., see \cite{AC:BloBluDat13,CCS:KogManBon17}), we use a Poisson arrival process to generate the sequence. The Poisson arrival process is parameterized by an arrival rate $T_u$ (hours), which encodes the expected time between consecutive login attempts $T_u = \mathbb{E}[t_{i+1}-t_i]$. The arrival process is memoryless, so the actual gap $t_{i+1}-t_i$  is independent of $t_i$. Since some users are more active than others, we pick a different arrival rate $T_u$ for each user $u$ where each $T_u$ is sampled uniformly at random from $\{ 12, 24, 24 \times 3, 24 \times 7, 24 \times 14, 24 \times 30\}$. The parameter $T_u = 12$ (hours) corresponds to users who login to their accounts twice per day on average, while the parameter $T_u = 24 \times 30$ corresponds to a user who visits the site once per month. We assume that users continue attempting to login for each user visit until they succeed or get locked out. 



\vspace*{-\baselineskip}
\vspace*{-\baselineskip}
\subsubsection{Simulating Users' Mistakes}\label{section:ExperimentDesign-subsection:SimulateUser-subsubsection:SimulateUserMistake} %Done
\vspace*{-\baselineskip}
The last component of our user model is a mechanism to simulate users’ honest mistakes during the authentication process. Our model relies upon recent empirical studies of password typos~\cite{CCS:CWPCR17,SP:CAAJR16} and additionally incorporates other common user mistakes, e.g., recall errors. The aforementioned studies show that roughly $7.5\%$ of login attempts are mistakes, and at least $68\%$ of them are (most likely) typos, i.e., within edit\deleted{ing} distance $2$ of the original passwords.  



Accordingly, in our simulation we set the mistake rate to be $7.5\%$, i.e., when simulating each login attempt, the user will enter the correct password with probability $92.5\%$. Otherwise, we simulate the user's error(s) --- either a recall error or a typo or both. In our simulations of user errors we first flip a biased coin to determine whether to simulate a typo ($68\%$) or a recall error ($32\%$). To simulate a recall error, we randomly select one of the user's five alternate passwords to model a user who forgot which of their passwords was associated with this particular account (the user may additionally misstype this password). When simulating different types of typos (captalization errors, substitution errors, insertion/deletion errors) we rely on empirical password typo data from  \cite{SP:CAAJR16,CCS:CWPCR17}.   We refer an interested reader to \lazyref{Appendix}{appendix:simulateMistakes} for a more detailed discussion of our mistake model, including a flow chart (see Figure~\ref{figure:flowChartTypo}) and more fine-grained typo statistics. If the login attempt is incorrect the simulated user will repeat the above process until s/he is successful or until the account is locked.

%Based on the statistics mentioned earlier, we simulate typos and recall errors with probability $68\%$ and $32\%$, respectively. To simulate a recall error, we randomly select one of the user's five alternate passwords to model a user who forgot which of their passwords was associated with this particular account. If the user recalls the wrong password, they might additionally miss-type it (with probability $0.075\cdot 0.68$).



{\bf Remark: } To study the throttling effects of $\DALock$, we do not simulate users who {\em completely} forget their passwords ( i.e., meaning that the probability of remember\added{ing} the correct password is non-zero during each login attempt) as these users will need to reset their passwords independently of the deployed throttling mechanism. In addition, we do not simulate a client device that automatically attempts to login on the user's behalf using a stored password. It may be desirable to have the authentication server stores the (salted) hash of the user’s previous password(s) to avoid locking the user's account in settings where a client device might repeatedly attempt to login with an outdated password incrementing both the hit-count $\hitCountThresholdOfU{u}$ and the strike count $\KOfU$. Alternatively, the authentication server could store an encrypted cache of failed login attempts using public-key cryptography. Each failed login attempt $pw_u' \neq pw_u$ would be encrypted with a public key $pk_u$ and stored on the authentication server. The encrypted cache could only be decrypted when the user authenticates with the correct password\footnote{Unlike the public encryption key $pk_u$, which would be stored on the authentication server, the secret key $sk_u$ would only be stored in encrypted form i.e., the server would store $c_u = \mathbf{Enc}_{K_u}(sk_u)$ where $K_u = \mathbf{KDF}(pw_u)$ is a symmetric encryption key derived from the user's password. }. The encrypted cache could be used as part of a personalized typo corrector~\cite{CCS:CWPCR17} and could also be used to avoid penalizing repeat mistakes~\cite{CCS:CWPCR17,EuroSP:THS19}. One potential downside to this approach is that the cache might inadvertently contain credentials from other user accounts, making cached data valuable to the attacker. More empirical studies would be needed to determine the risks and benefits of maintaining such a cache.


%Wuwei: Below is mentioned previously.


%We remark that we do not attempt to simulate a user who completely forgets his password. Of course, we expect that this will occasionally happen in reality. However, we observe that a user who forgets his password will {\em always} need to reset it regardless of the throttling mechanism adopted by the authentication server.













% !TEX root = main.tex

\subsection{Modeling the Authentication Server}\label{section:ExperimentDesign-subsection:SimulateServer} %Done

We model an authentication server running $\DALock$ with various parameters $\strikeThreshold$ and $\hitCountThreshold$ for the strike count and hit count. Each time a user $u$ (or attacker pretending to be $u$) attempts to login the authentication server updates the parameters $\hitCountThresholdOfU{u}$ and $\strikeThresholdOfU{u}$ accordingly following the $\DALock$ mechanism. We remark that when $ \hitCountThreshold = \infty$ that the authentication server is running the classical $ \strikeThreshold$-strikes lockout policy. To deploy $\DALock$ with a finite hit-count parameter $ \hitCountThreshold$ an authentication server needs to use a frequency oracle to update the hit count after each incorrect login attempt.  In this work we consider two concrete approaches the authentication server might adopt: (differentially private) Count Sketch estimator and Password Strength Models. We use $\EstimateP{pw}{\Estimator}$ to denote the estimated popularity (probability) of a password $pw$ using the estimator $\Estimator$ e.g., given a Count-Sketch $\sigma$ we would use  $\EstimateP{pw}{\sigma} = \frac{\mathbf{Estimate}(pw,\sigma)}{\mathbf{TotalFreq(\sigma)}}$. We remark that the authentication server might (optionally) chose to ban overly popular passwords to flatten the password distribution to protect user accounts against online attackers \cite{HTS:SchHerMit10}. If the authentication server adopts such policy, then the frequency oracle would need to be adjusted accordingly to model the new password distribution.



\subsubsection{Differentially Private Count Sketch Estimator} 

The first instantiation of $\EstimateP{\cdot}{\cdot} $ we consider is to build a Count Sketch Estimator $\sigma_{\SampledData{\AllUser}} = \Add{\SampledData{\AllUser}}{\sigma} $ from our dataset $\SampledData{\AllUser} $ of user passwords. To build a Count Sketch in practice the authentication server would update the Count Sketch with the new password each time a user registers \footnote{The Count Sketch instantiations we consider would also support a Remove operation which would allow the authentication server to handle password updates efficiently}. There are several issues to consider when deploying the Count Sketch estimator: memory efficiency, privacy, sample size and accuracy. 


\textbf{Memory Efficiency} We instantiate the Count Sketch with parameters $d=5$ and $w=10^6$ so that the entire data structure requires just $20$ MB of space which easily fits in RAM. 


\textbf{Privacy} As we discussed earlier one concern about storing a Count Sketch $\sigma_{\SampledData{\AllUser}} $ on the authentication server is that an offline attacker might steal this file and use the data-structure to help identify user passwords. For example, if our user John Smith selects (resp. does not select) the password ``J.S.UsesStr0ngpwd!'' then we would expect that the true frequency of this password is $\TrueFInD{pw}{\SampledData{\AllUser} }=1$ (resp. $\TrueFInD{pw}{\SampledData{\AllUser} }=0$). If the Count Sketch estimator is overly accurate then the attacker would be able to learn that one user (most likely John Smith) picked this password. Without a way to address these privacy concerns an organization might be understandably wary to deploy a Count Sketch estimator.


To address these privacy concerns we consider an $\epsilon$- differentially private estimator $\sigma_{dp}$ = \textbf{DP($\epsilon,\sigma$)} in our experiments. During initialization we add Laplace noise to each of the cells in the Count Sketch where the noise parameter scales with $d/\epsilon$. In our above example, differential privacy ensures that --- up to a multiplicative advantage $e^{\epsilon}$ --- an attacker cannot use the count sketch to distinguish between a dataset in which John Smith did (resp. did not) pick the password ``J.S.UsesStr0ngpwd!' We remark that lower values of $\epsilon$ correspond to stronger privacy guarantees e.g., we use $\epsilon=\infty$ to denote the case with no differential privacy guarantees. In most of our experiments we use a small privacy parameter $\epsilon=0.1$ which is much smaller than the privacy parameters used in most prior deployments of differential privacy e.g.,  \cite{NDSS:BloDatBon16,AppleDPTeam,CCS:ErlPihKor14}. 


\textbf{Sample Size and Accuracy} In general the accuracy of a Count Sketch increases with the size of the password dataset. Suppose that the organization does not have millions of users or the that the sample size is decreased because the organization allows users to ``opt-in'' to the (differentially private) count sketch. One natural question is whether a smaller organization would be able to deploy a Count Sketch to obtain reliable frequency estimates. We investigate this question by subsampling smaller datasets to train the Count Sketch. Given a set $\AllUser$ of $N$ users we use $\AllUser_{r\%}$ to denote a randomly subsampled set of $r\%$ of users. We use $\SampledData{\AllUser_{r\%}}$ to denote the corresponding subsampled password dataset $\sigma_{r\%} = \Add{\SampledData{\AllUser}}{\sigma} $ to denote the Count Sketch trained on the subsampled data. The question is whether $\sigma_{r\%}$ can be as effective as $\sigma$ for deploying $\DALock$. 


In our experiments we consider the following sampling rates: 1\%, 5\%, and 10\%. We find that even when $r=1\%$ the Count Sketch $\CountSketch$ trained on $\SampledData{\AllUser_{1\%}}$ is sufficiently accurate --- even if we additionally add Laplace noise to preserve $\epsilon=0.1$-differential privacy. 


%Let $\sigma_{r\%}$ be the Count Sketch trained based on $\AllUser_{r\%}$, r\% of the users,  who choose to participate. The question is whether $\sigma_{r\%}$ can be as effective as $\sigma$ for deploying $\DALock$. To investigate it's deployability, we include $\sigma_{r\%}$ constructed by $\AllUser_{r\%}$ as part of our experiments. 


%Count Sketch Estimator $\sigma$ is the first approach for implementing frequency oracle. A standard Count Sketch $\sigma$ trained based on input dataset $\SampledData{\AllUser}$ (as described in \textbf{Section}~\ref{section:Prelinmaries-CountSketch}) can be memory-efficient and accurate. However, $\sigma$ may not be obtainable or usable due to various reasons in real life scenarios (e.g. information protection law). Therefore, we also consider the following types of $\CountSketch$ estimators.


%\textbf{Differential Private Count Sketch}

%Privacy leakage risks exist if one directly deploys $\DALock$ with standard $\sigma$. One sounding solution is to adopt differential privacy to reduce the surface of vulnerabilities; however, it's questionable whether Count-Sketch based $\DALock$ can still be effective when privacy budget is limited. e.g. $\epsilon = 0.1$. To assess its effectiveness under such situation, we also include differential private Count-Sketch estimator $\sigma_{dp}$ = \textbf{DP($\epsilon,\sigma$)} in the experiments. In particular, we are interested in low privacy budget scenarios so data curators can periodically update $\sigma_{dp}$ without significant cumulative privacy loss.




%\textbf{Building Count Sketch with Low Participation Ratio}

%There are multiple incentives to train a Count Sketch on a small dataset. For instance, new business may not have an enormous number of users at beginning. For mature ones, one challenge data curator may face is low participation in password statistics sharing. Human generated passwords can contain sensitive information such data of birth, therefore it's possible that not everyone is willing to opt-in. 




\textbf{Count Sketch with Banlists} In our simulations we also consider an authentication server that bans the most popular $B=10^4$ passwords in a dataset to help flatten the password distribution and protect users against online attacks. Theoretical analysis indicates that directly banning the most popular passwords is the most effective way to increase the minimum entropy of the password distribution~\cite{BKPS:ACMEC13}. We remark that one additional benefit of using a Count Sketch data structure is that it can be used to help implement this type of policy i.e., if a user attempts to register with password $pw$ and $\EstimateP{pw}{\sigma}$ is already too high then the user will be required to pick a different password~\cite{HTS:SchHerMit10}.


We evaluate the performance of $\DALock$ in the presence of banlists. Recall that we let $\SampledData{\AllUser, B}$ denote the dataset $\SampledData{\AllUser}$ with the $B$ most common passwords removed following the normalized probabilities model of ~\cite{BKPS:ACMEC13} to model how affected users will update their passwords in response to the banlist. In particular, we assume users who are affected by the policy will pick a new passwords following the empirical distribution induced by $\SampledData{\AllUser, B}$. We then train the Count Sketch on the updated dataset i.e., $\sigma_{-B} = \text{Add}(\SampledData{\AllUser, B})$ as follows. 


% The core idea behind $\DALock$ is punishing attempts with over popular passwords. One natural following question is what happens if the distribution is not so skewed? One way to achieve such distribution is via banning over popular credential\cite{HTS:SchHerMit10}. Recent studies show\cite{BKPS:ACMEC13,HTS:SchHerMit10} that password composition policies can flaten distributions and discourage attackers. $\DALock$ punishes attempts on over popular credentials while there is no such password under this circumstance. It is worthwhile to test if $\DALock$ can still outperform traditional throttling mechanism. To conduct this type of experiment we implement $\DALock$ with Count Sketch $\sigma_{-b} = \text{Add}(\SampledData{-b})$ as follows. Let b the number of banned passwords, we first construct $\SampledData{-b}$ via banning top b passwords in $\SampledData{\AllUser}$. We assume users who are affected by the policy will pick a new password based on the distribution of remaining passwords in this process. After that, we construct $\sigma_{-b}$ based on the obtained dataset $\SampledData{-b}$



\subsubsection{Frequency Oracle from Password Models}

As we previously discussed there are several reasons why an organization might prefer not to use a Count Sketch for frequency estimation e.g., privacy concerns or limited sample size. An alternative is to instantiate the frequency oracle with a password model. This could be a heuristic password strength meter, a more sophisticated model based on Neural Networks, Probabilistic Context Free Grammars or Markov Models or an empirical estimate based on Hashcat. The primary advantage to this approach is that the model can be deployed immediately even before an organization has any users and there are no privacy concerns. 


We adopted the $\ZXCVBN$ password strength meter~\cite{USENIX:Wheeler16} as prior empirical studies  demonstrate that it is one of the most accurate password strength meters \cite{CCS:GolDur18}. We used the Password Guessing Service \cite{USENIX:USBCCKKMMS15} to obtain guessing numbers for Neural Network, PCFG, Hashcat, and Markov Models ---  we also considered the  minimum guessing number across all four models as suggested in \cite{USENIX:USBCCKKMMS15}. For example, if a password $pw$ had guessing number $g$ we might estimate that $\EstP{pw_i} =1/g$. One challenge that we needed to address was that the estimates we obtain do not always yield a probability distribution e.g., for $\ZXCVBN$ we have $\sum_{i=1}^{10000}\EstP{pw_i} \gg 1$ where $i$ ranges over the top $10^4$ passwords in the dataset. Thus, before deploying the frequency estimator in $\DALock$ we renormalized our estimates so that $\sum_{i=1}^{10000}\EstP{pw_i} =1$. 


% In this work we also seek alternatives to $\CountSketch$ which does not require collecting users' passwords from the server. Literature\cite{CCS:GolDur18} shows that some of them can adequately compute the strength of weak credentials. Therefore, we implement $\DALock$ with the following frequency oracles and tested their performance: $\ZXCVBN$\cite{USENIX:Wheeler16}, Neural Network, PCFG, Hashcat, and Markov Model ( the later four are based on estimation of PGS\cite{USENIX:USBCCKKMMS15}). 

% Implementing frequency oracle by this approach can be a challenging task as they output (strength) scores for passwords in lieu of popularities. To overcome the issue, we define the popularites of password $pw$ to be its estimated score over the total score of top 20,000 passwords. i.e $\EstP{pw} = \frac{\EstP{pw}}{\sum_{i=1}^{20000}\EstP{pw_i}}$.


% !TEX root = main.tex
\subsection{Modeling the Attacker}\label{section:ExperimentDesign-subsection:SimulateAttacker} % Done
The final component of our simulation is a model of the attacker. We take a conservative approach and model an untargetted attacker with complete knowledge of the password distribution. Following Kerckhoff's principle we also assume that the attacker has access to the complete description of the $\DALock$ mechanism. In particular, for any password $pw$ we assume that the attacker knows both the true probability $ \TrueP{pw}$ and the estimated probability $\EstP{pw}$.  Finally, we also assume that the attacker is given the complete sequence of login times $t_1^u \leq t_2^u \leq  \ldots \leq 24 \times 180$ for each user $u$ over a 180 day time span as well as the outcome of each login attempt e.g., at time $t_i^u$ user $u$ will succeed after 2 incorrect guesses. 

{\bf Remark:} We conservatively aim to overestimate the capabilities of an untargetted online attacker. In practice, the online attacker will be able to able to approximate $ \TrueP{pw}$  and $\EstP{pw}$ overtime by interacting with the $\DALock$ server e.g., by setting up dummy accounts to test many times he can submit a particular incorrect guess without exceeding the hit count. Similarly, the attacker would not necessarily know the exact login times for a user, but this conservative assumption makes it feasible to precisely characterize the optimal behavior of an attacker. In practice, an online attacker might wait several days in between guesses to avoid accidently locking the user's account based on the number of consecutive incorrect login attempts. 

\subsubsection{Optimizing Attack Strategies} %Done
The goal of the attacker is to maximize the probability of cracking each password within the fixed 180-day time span. For example, the attacker might try to find popular passwords $pw$ where the ratio $\EstP{pw}/\TrueP{pw}$ is small so that the increased hit count is smaller than intended. We formalize the attacker's optimal strategy in terms of the \textsf{Password Knapsack} problem $(\PK)$. Unsurprisingly, the password knapsack problem turns out to be $\NP$-hard (as we prove in the appendix), but there are several heuristic algorithms the $\Adversary$ can use which yield nearly optimal strategies in practice. 

Recall that we assume that the $\Adversary$ has perfect knowledge of the distribution and probability estimates for each password $pw$. We also assume $\Adversary$ knows the $\DALock$ security parameters $\strikeThreshold$ and $\hitCountThreshold$. Furthermore, for each user $u$ we assume that the attacker is given the complete sequence of login times $t_1^u \leq t_2^u \leq  \ldots \leq 24 \times 180$ for each user $u$ over a 180 day time span as well as the outcome of each login attempt e.g., at time $t_i^u$ user $u$ will succeed after 2 incorrect guesses. In particular, at any point in time $t < 24\times 180$ the attacker can infer the current strike threshold and hit count threshold for any user $u$. We denote by $\strikeThresholdOfU{u,t}$ (resp. $\hitCountThresholdOfU{u,t}$) the strike (resp. hit count) threshold  for user $u$ at time $t$ assuming that the attacker does not submit any of his own guesses. 

Supposing that the attacker wishes to avoid locking down the user's account before time $t$ the cumulative (estimated) probability of all guesses submitted before that time should be at most $\hitCountThresholdOfU{u,t}':=\hitCountThreshold- \hitCountThresholdOfU{u,t}$. Similarly, we let $M(t)$ denote the maximum number of guesses that the attacker can sneak in over the first $t$ hours without locking down the account i.e., because $\strikeThresholdOfU{u,t'}  \geq \strikeThreshold$ at some time $t' \leq t$. 

Fixing the time parameter $t$ the attacker’s goal is to find a subset $S_t \subseteq \AllPassword$ of $M(t)$ passwords to check such that 
\begin{equation} \label{eq:attackerConstraint}
	\vspace{-0.2cm} 
	\sum_{pw \in S_t} \EstP{pw} \leq \hitCountThresholdOfU{u,t}' \ . \vspace{-0.1cm} 
\end{equation}
After checking the passwords in $S_t$ the attacker can still check one more password $pw_{hold} \not\in S_t$ before the account is locked down. Given a set $S_t$ and a holdout password $pw_{hold} \not\in S_t$ the probability that the attacker succeeds is 
\begin{equation}\vspace{-0.2cm} \label{eq:attackerSuccess} \TrueP{pw_{hold}} + \sum_{pw \in S_t}\TrueP{pw} \ . \vspace{-0.1cm} \end{equation}

Thus, the goal of the attacker is to find a subset $S_t$ of size $|S_t| \leq M(t)$ maximizing his success rate (eq \ref{eq:attackerSuccess}) subject to the constaint in  equation \ref{eq:attackerConstraint}.

\mypara{Password Knapsack Problem}  Given a password dictionary \\$\{pw_1, \ldots, pw_n\}$ we formally define the \textsf{P}assword \textsf{K}napsack($\PK$) problem as the following integer program with indicator variables $s_i \in \{0,1\}$ and $l_i=\{0,1\}$ for each password $pw_i$. The attackers goal is to select a holdout password and a separate subset of $M$ ($=M(t)$) passwords with total `weight' (estimated probability) at most $\hitCountThreshold'$ ($= \hitCountThresholdOfU{u,t}'$) 

$$
\begin{array}{crl}
	&\max {\displaystyle{\sum_i {(s_i + l_i) \cdot \TrueP{pw_{i}}}}} \\
	subject\ to, &\\
	&\sum_i{s_i \cdot \EstimateP{pw_i}{\sigma}) \le \hitCountThreshold'} \\
	&\sum_i s_i \le M\\
	&\sum_i l_i \le 1\\
	&\forall i~ l_i + s_i \le 1\\
	where,\\
	& \forall i, s_i, l_i \in \{0,1\}
\end{array}
$$
Intutively, setting $s_i$ = 1 means $pw_i$ is selected to be placed in the ``password knapsack" $S\subseteq \AllPassword$, i.e. to be used for dictionary attack. Setting $l_i=1$ indicates that password $pw_i$ is used as holdout password. This is equivalent to the following optimization problem. The constraints ensure that $|S| \leq M$ and we pick exactly one holdout password that is not already in $S$. 

\mypara{Solving the \textsf{P}assword \textsf{K}napsack} To maximize the number of cracked passwords an online attacker can compute $M(t)$ and $\hitCountThresholdOfU{u,t}':=\hitCountThreshold- \hitCountThresholdOfU{u,t}$ for each time $t \leq 24 \times 180$ and solve the corresponding \textsf{P}assword \textsf{K}napsack problem. Given optimal solutions $(pw_{hold,t}^*, S_t^*)$ for each time $t$ the attacker will pick the solution that maximizes the number of cracked passwords as in equation \ref{eq:attackerSuccess}. We remark that the calculations above need to be repeated for each different user $u$ since the values $M(t)$ and $\hitCountThresholdOfU{u,t}'$ may vary due to different visitation schedules.







\mypara{Solving Password Knapsack}  Unfortunately, the \textsf{P}assword \textsf{K}napsack problem is $\NP$-hard as we prove in \textbf{Theorem}~\ref{appendix:ProofOfPasswordKnapsack} in the Appendix via a straightforward reduction from Subset Sum. In all of instances we considered we found that the optimal choice for the holdout password was simply $pw_1$ the most likely password in the distribution. Once we fix our holdout password our problem reduces to the two-dimensional knapsack problem. We remark that $\PK$ can be viewed as a two-dimensional knapsack problem. 

Assuming $P\neq NP$ the two-dimensional knapsack problem does not even admit a polynomial time approximation scheme ($\PTAS$) \cite{kulik2010there} in contrast to the regular knapsack problem which has fully polynomial time approximation scheme ($\FPTAS$)). Thus, we consider two heuristic approaches to solve the password knapsack problem:  $\mathsf{D}$antizig's $\mathsf{A}$lgorithm $\mathsf{B}$ased\cite{Dan:OR57} approach (\DAB) and $\mathsf{F}$easible $\mathsf{M}$ost $\mathsf{P}$romising $\mathsf{P}$assword $\mathsf{F}$irst approach(\FMPPF).

$\DAB$ (\textbf{Algorithm}~\ref{algorithm:Dantizig}, Appendix) sorts the remaining passwords $\mathcal{P}_{\tilde{\Pi}} = \{pw_2, \ldots\, pw_n\}$ based on the ratios $\frac{\TrueP{pw_i}}{\EstP{pw_i}}$ and select candidates based on the sorted order until we either select $M$ passwords or until selecting another password would exceed our capacity $\hitCountThreshold'$. $\FMPPF$ (\textbf{Algorithm}~\ref{algorithm:FMPPF}, Appendix) sorts the remaining passwords based on the true probability $\TrueP{pw_i}$  and simply selects password $pw$ in sorted order until we either select $M$ passwords or until selecting another password would exceed our capacity $\hitCountThreshold'$.  We discuss the advantages and disadvantages to both heuristics in the appendix. Intuitively, $\FMPPF$ (resp. $\DAB$) will perform better when $M$ (resp. $\hitCountThreshold'$) is the limiting constraint.

We found that $\FMPPF$ generally performs better than $\DAB$ despite of its simplicity. In addition, our simuation shows that $\FMPPF$'s performance is close to optimal. Practically speaking, one generally expect $\EstP{pw_i} \approx \TrueP{pw_i}$ especially when $pw_i$ is a popular password. In such case, $\DAB$ can hardly gain advantages from underestimation. Furthermore, imagine one bucket passwords by probability ranges, there are plenty of passwords in each bucket. Intuitively, picking passwords ordered by $\TrueP{pw_i}$ should produce an (almost) optimal solution (quickly). Thus, we choose to present the results based on $\FMPPF$ approach.







% !TEX root = main.tex
\section{Experimental Results}\label{section:experimentalresult} %Done
We empirically evaluated the performance of $\DALock$ under a variety of scenarios. During each simulation we had $10^6$ honest users register with an authentication server running $\DALock$ and login over a period of $180$ days. To analyze usability, we ran the simulations without an online password attacker and measured the unwanted lockout rate i.e., the fraction of user accounts that were locked due to honest mistakes. To analyze security, we added an untargetted online attacker to the simulation and measured the fraction of user passwords that the attacker cracked. 


The results of our simulations are summarized in \textbf{Figures}~\ref{figure:dictionaryAttack9.375},~\ref{figure:usability9.375}, ~\ref{figure:dictionaryAttack7.0}, ~\ref{figure:usability7.0},~\ref{figure:dictionaryAttackPrune}, and ~\ref{figure:usabilityPrune}\footnote{We included additional experimental results in \textbf{Appendix} ~\ref{appendix:experimentalResults} for interested readers. }.  The first four figures evaluate the security (\textbf{Figures}~\ref{figure:dictionaryAttack9.375} and~\ref{figure:dictionaryAttack7.0}) and usability (\textbf{Figures}~\ref{figure:usability9.375} and~\ref{figure:usability7.0}) in the absence of a banlist.  The last two figures evaluate the usability (\textbf{Figure}   ~\ref{figure:usabilityPrune} ) and security (\textbf{Figure} ~\ref{figure:dictionaryAttackPrune} ) of $\DALock$ when the authentication server bans the top $B=10^4$ passwords in our dataset. The X-axis of each plot represents the time span over 180 days. And the Y-axis represents percentage of compromised users (unwanted locked out rate) for security (usability) experiments.

\mypara{Implementation Details}  In each of our implementations $\DALock$ we instantiated $\strikeThreshold=10$ using hit count parameters $\hitCountThreshold \in \{ 2^{-7.0}, 2^{-9.375}\}$ (no banlist) and $\hitCountThreshold=2^{-11}$ (with banlist). In each batch of experiments we used one of our three password datsets $\SampledData{\AllUser}$ (RockYou, Yahoo!, or LinkedIn) to define our password distribution and we instantiated the frequency oracle using (1) a variety of password models including $\ZXCVBN$~\cite{USENIX:Wheeler16}, Hashcat, Markov Models, PCFGs and Neural Networks~\cite{USENIX:USBCCKKMMS15}\footnote{We relied on the Password Guessing Service to obtain cracking numbers for Hashcat, Markov Models, PCFGs and Neural Networks~\cite{USENIX:USBCCKKMMS15}. We found that the Neural Network model failed to crack many weak passwords in our datasets as it was configured not to guess short passwords. As such we did not directly use Neural Networks to implement our frequency oracle. However, the Neural Network guessing numbers are included in our Min-all frequency oracle which uses the minimum guessing number over all models.} and (2) an $\epsilon$-differentially private count sketch trained on a subsample $\SampledData{\AllUser_{r\%}}$ of containing $r\%$ of the original dataset $\SampledData{\AllUser}$ with $r \in \{1\%, 5\%, 10\%, \mathtt{All}\}$. When instantiated with banlist we used the dataset $\SampledData{\AllUser, B}$ instead of $\SampledData{\AllUser}$ i.e., we  removing the $B=10^4$ most common passwords.




%In each simulation we instantiated the frequency oracle 



\mypara{Baseline} We used the classical $3$-strike mechanism  and the $10$-strike mechanisms (recommend by Brostoff et. al \cite{brostoff2003ten} to improve usability) as a baseline for comparison. We remark that this is equivalent to $\KPsiDALock{3}{\hitCountThreshold=\infty}$ and $\KPsiDALock{10}{\hitCountThreshold=\infty}$ respectively. Our results suggest that one can improve {\em both} security and usability by replacing the classic 3-strike throttling mechanism with $(10,\hitCountThreshold)-\DALock$. Our results demonstrate that $\KPsiDALock{10}{\hitCountThreshold}$ greatly outperforms the classic $10$-strikes throttling mechanism without significant usability loss -- from a usability standpoint decreasing $\hitCountThreshold$ can only increase the unwanted lockout rate. We discuss our findings in more detail below.






\subsection{Usability}\label{section:ExperimentResult-usability} %done
\textbf{Figures} ~\ref{figure:usability9.375} and ~\ref{figure:usability7.0} highlight the usability of $\DALock$ in the absence of a banlist with hit count parameters $\hitCountThreshold = 2^{-9.375}$ and $\hitCountThreshold=2^{-7.0}$. When $\hitCountThreshold=2^{-7.0}$ we find that $\DALock$ {\em always} outperforms the classical $3$-strikes mechanism regardless of how the frequency oracle is instantiated. When $\hitCountThreshold=2^{-9.375}$ $\DALock$ still outperforms the classical $3$-strikes mechanism when instantiated with a count-sketch --- even when we train on just $r=1\%$ of the dataset and add Laplace noise to achieve $\epsilon=0.1$-differential privacy. When we instantiate $\DALock$ password models the results were mixed e.g., $\ZXCVBN$ and Hashcat had superior usability while Markov Models and Probabilistic Context Free Grammars performed poorly. 

\textbf{Figure} \ref{figure:usabilityPrune} highlights the usability benefit of banning the top $10^4$ passwords. While a few users might be inconvenienced during the password registration our simulations indicate that the unwanted lockout rates for $\DALock$ are greatly reduced even when we adopt a stricter $\hitCountThreshold = 2^{-11}$. Intuitively, the banlist allows us to avoid locking out users who select and overly popular password as one of their five alternate passwords for other accounts. We remark that when we instantiate $\DALock$ with a count sketch that the usability results are virtually identical to the $10$-strikes policy --- even if we use a $\epsilon=0.1$ -differentially private count sketch trained on just $1\%$ of the data.



%\mypara{Count Sketch Based $\DALock$ beats 3-strike mechanism}
%Our simulation results suggest that one can replace 3-strike mechanism by $\KPsiDALock{10}{\Psi}$ for various settings of $\Psi$ to improve usability. Both $\KPsiDALock{10}{2^-7}$ and $\KPsiDALock{10}{2^-9.375}$ have lower unwanted locked out rate compare to 3-strike mechansim across all datasets. For example, $\KPsiDALock{10}{2^-7}$ results in less than 1\% of locked out rate on RockYou dataset. To observe such effect, one can compare curves $\CountSketch$-All(k:10,$\hitCountThreshold:2^{-9.375},\epsilon: \infty$), $\CountSketch$-All(k:10,$\hitCountThreshold:2^{-7.0},\epsilon: \infty$) and 3-strike($\epsilon:\infty$) from \textbf{Figure}~\ref{figure:usability9.375} and ~\ref{figure:usability7.0} . 

\mypara{10-strike Mechanism is user friendly} Brostoff et. al \cite{brostoff2003ten} proposed that one should replace 3-strike mechanism with 10-strike mechanism to achieve higher usability. Our simulation results clearly align with their recommendations. Based on our plots, 10-strike mechanism results in unwanted locked out rate close to zero. However, deploying this mechanism also has a high security cost as indicated by our simulations. 





%\textbf{Figure}~\ref{fig:figure:attacker_no_dp} in \textbf{Appendix}~\ref{appendix:security_experiment}

\begin{figure*}\label{key1}
	\includegraphics[width=\linewidth, height = 4.5cm]{Figures/Experiments/Attacker/DictionaryAttack.png}
	\caption{Security Measurement of $\DALock$ - $\hitCountThreshold = 2^{-9.375}$ }\label{figure:dictionaryAttack9.375}
	\includegraphics[width=\linewidth, height = 4.5cm]{Figures/Experiments/Utility/Utility.png}
	\caption{Usability Measurement of $\DALock$ - $\hitCountThreshold = 2^{-9.375}$ }\label{figure:usability9.375}
	\includegraphics[width=\linewidth, height = 4.5cm]{{Figures/Experiments/Attacker/DictionaryAttack_7.0.png}}
	\caption{Security Measurement of $\DALock$ - $\hitCountThreshold = 2^{-7.0}$ }\label{figure:dictionaryAttack7.0}
	\includegraphics[width=\linewidth, height = 4.5cm]{{Figures/Experiments/Utility/Utility_7.0.png}}
	\caption{Usability Measurement of $\DALock$ - $\hitCountThreshold = 2^{-7.0}$ }\label{figure:usability7.0}
\end{figure*}

\begin{figure*}\label{key2}
	\includegraphics[width=\linewidth, height = 4.5cm]{Figures/Experiments/Attacker/DictionaryAttackPrune.png}
	\vspace{-0.8cm}
	\caption{Security Measurement of $\DALock$ - $\hitCountThreshold = 2^{-11}$ (Banning top $B=10^4$ passwords)  }\label{figure:dictionaryAttackPrune}
	\includegraphics[width=\linewidth, height = 4.5cm]{Figures/Experiments/Utility/UtilityPrune.png}
	\vspace{-0.8cm}
	\caption{Usability Measurement of $\DALock$ - $\hitCountThreshold = 2^{-11}$ (Banning top $B=10^4$ passwords)}\label{figure:usabilityPrune}
	\vspace{-0.6cm}
\end{figure*}



\subsection{Security Results} \label{section:ExperimentResult-security} %done
\textbf{Figures}~\ref{figure:dictionaryAttack9.375} and ~\ref{figure:dictionaryAttack7.0} evaluate the security of $\DALock$ in the absence of a banlist with hit count parameters $\hitCountThreshold = 2^{-9.375}$ and $\hitCountThreshold=2^{-7}$ respectively. For reference we also plotted the line $\hitCountThreshold + \TrueP(pw_1)$ which constitutes a theoretical upper bound on the attacker success rate when $\DALock$ is instantiated with a perfect frequency oracle. We found that $\DALock$ always outperforms even the stricter $\strikeThreshold=3$-strikes policy under {\em all} instantiations of the frequency oracle (excluding Hashcat with $\hitCountThreshold=2^{-7}$). For example, the attacker cracked roughly 4\% of users accounts when facing 3-strike mechanism (LinkedIn + RockYou) compared with $2\%$ when deploying $\DALock$  with a differentially private count sketch.

\textbf{Figure}~\ref{figure:dictionaryAttackPrune} highlights the advantage of banning the most popular $B=10^4$ passwords. We remark that $\DALock$ always outperformed the $\strikeThreshold=10$-strikes policy. We also found that the $\strikeThreshold=3$-strikes policy and $\DALock$ (with a differentially private Count Sketch) were both {\em highly effective} at protecting user accounts with a compromised rate is about close to zero. ( $\approx$ 0.092\% for 3-strike and  $\approx$ 0.048\% for differential private Count Sketch).

\subsection{Summary and Discussion}\label{sec:experiment_summary}
We find that $\KPsiDALock{10}{\Psi}$ offers a superior security/usability tradeoff to the classical $\strikeThreshold$-strikes mechanism. Our experiments also highlight the security {\em and} usability benefits of banning overly popular passwords. We found that $\DALock$ can be reasonably instantiated with password strength models such as $\ZXCVBN$, Markov Models, Probabilistic Context Free Grammars and Neural Networks. However, we obtain the best security/usability tradeoffs when we ban the most popular passwords and when we instantiate the $\DALock$ frequency oracle    with a differentially private count sketch. We found that the count sketch can be reliable trained from a smaller subsample containing just $1\%$ of the dataset, and even if we add enough noise to preserve $\epsilon-0.1$-differential privacy (strong privacy). This is promising news for a smaller organization that is considering deploying $\DALock$.

%Our results demonstrate that one can easily replace 3-strike mechanism with a properly configured $\KPsiDALock{10}{\Psi}$. In particular, it is easier to find an effective $\Psi$ for Count-Sketch based $\DALock$ compare to other frequency oracles. Implementing $\DALock$ with other frequency oracles is possible but requires more efforts on engineering the parameters. In addition, we show that $\sigma_{r\%}$ and $\sigma_{dp}$ are as effective as standard Count Sketch estimator $\sigma$ which makes deployment an easier task. Finally, we discovered that one can effectively eliminate dictionary attack by flatten password distribution (e.g. banning popular passwords). In this scenarios, one can still deploy $\DALock$ to achieve better utility. 



% !TEX root = main.tex
\mypara{Limitations}\label{sec: Limitations}
Our empirical {usability and} security results are all based on simulations. While we aim to model the authentication server, users, and a powerful attacker, there will inevitably be some differences between the simulated/real-world behavior of the attacker/users. We also remark that our simulations do not model the behavior of targeted attackers. Extending $\DALock$ to protect against targeted attackers is an important research question that is beyond the scope of the current paper. {Another future direction of study is to conduct a longitudinal user studies to confirm the ecological validity of the simulated usability results.} Finally, we remark that larger organizations might distribute the workload across multiple authentication servers. In this case maintaining a synchronized state $(\strikeThresholdOfU{u}, \hitCountThresholdOfU{u})$ for each user $u$ could be challenging. To address this challenge, it may be necessary to define a relaxation of our $\DALock$ mechanism where the states $(\strikeThresholdOfU{u}, \hitCountThresholdOfU{u})$ on each authentication server are not always assumed to be perfectly synchronized. 

{
\mypara{Locking Accounts vs Blocking IPs} In our simulated evaluation of $\DALock$ we assume that each user {\em account} $u$ is locked if the hit count $\hitCountThresholdOfU{u}$ exceeds the threshold $\hitCountThreshold$ (or if the consecutive strike threshold $\strikeThreshold$ is reached). An alternative (more lenient) implentation of $\DALock$ would instead maintain the state $(\strikeThresholdOfU{u,ip}, \hitCountThresholdOfU{u, ip})$  for each distinct user/IP pair $(u,ip)$ where $\hitCountThresholdOfU{u,ip}$ (resp. $\strikeThresholdOfU{u,ip}$) tracks the total hit count (resp. consecutive incorrect guesses) for all guesses submitted from the IP address $ip$ against user $u$. } { Under this alternate approach we could block a (malicious) ip address from attemting to login to account $u$ once $\hitCountThresholdOfU{u, ip}$ exceeds the threshold   $\hitCountThreshold$. One advantange of this approach is that it is less vulnerable to denial of service attacks and we are less likely to lockout the legitimate user who will (most likely) have a different IP address. Furthermore, this approach may be easier to implement in a distributed setting as the servers do not need to synchronize the state $(\strikeThresholdOfU{u}, \hitCountThresholdOfU{u})$ for each user --- instead each authentication server would independently maintain  the value $(\strikeThresholdOfU{u,ip}, \hitCountThresholdOfU{u, ip})$ for IP addresses in its service area. On the downside blocking individual IP addresses instead of accounts allows an distributed online attacker to launch coordinated attacks from multiple different IP addresses (e.g., through botnets) increasing the the risk to each user account. }

 





%\mypara{Other Attacks} $\DALock$ assumes that adversaries perform rational online dictionary attacks. Therefore, it can be vulnerable against other methods such as targetted attack and denial-of-service attacks. We leave this as an important research topic for the furture studies.

%\mypara{Protecting Weakest User} No matter how small $\Psi$ is deplyed, $\Adversary$ can always compromise the weakest users under the general framework of $\DALock$ since thresholds are maintained \textit{after} verification. Without the help of stricter authentication protocols such as 2FA, $\DALock$ alone is not able to reduce such lower bound.

%\mypara{Password Distribution}
%$\DALock$ assumes that the stored password distribution is close to the real one. Firstly, $\Adversary$ can maliciously register enornmous amount of accounts to lower the popularity of actual frequent passwords. The account creation process shall be carefully auditted to overcome the issue. Secondly, Since users can change their passwords at any time, one should update the Count-Sketch periodically (such as every 6 month).
%\mypara{Simulating Users' mistake} In this work, we simulate user's mistakes by considering two types of errors: 1) Typos, and 2) Entering a different password based on the work of Chatterjee et al\cite{CCS:CWPCR17}. There can be a gap between real world scenarios and our simulations. For example, users who use malfunctioning keyboards are likely to repeat their mistakes. 

%\jeremiah{A bit awkward to have two comparisons}
%\mypara{$\textsf{\textbf{DALock}}$ and StopGuessing\cite{EuroSP:THS19}}

%As mentioned in \textbf{section}~\ref{sec: relatedwork}, both $\DALock$ and StopGuessing utilize the distribution of passwords to defend against dictionary. In this part, we further discuss the difference of these two works.

%\textbf{Different Passwords Distribution} Firstly, $\DALock$ and StopGuessing collects different distributions. $\DALock$ uses true password distriubtion (perturbed by differential privacy) to estimate popularity of password. On the other hand, StopGuessing uses the distribution of \textit{failed attempts}. Maliciously altering the former distribution is a challenging task for attackers because significantly many accounts have to be created. Perturb the distribution stored by StopGuessing is eaiser as it only requires spamming attempts.


%\textbf{Throttling Threshold} $\DALock$ maintains a \textit{global fractional} hit count threshold $\Psi$ for rate limiting. On contrary,  StopGuessing implicitly maintains interger threshold for each user u based on the strength of password $p_u$. For $\DALock$, service provider can intuitively $\Psi$ based on the risk one is willing to take. For StopGuessing, it is less straight-forward to setup all the parameters such as penality functions.

%\textbf{Against Dictionary Attack} $\DALock$ and StopGuessing can easily be integrated  to provide a solid defense against dicitonary attack. $\DALock$ offers concrete account protection because attackers is subject to limited chance of success. This is not true for StopGuessing if the adversary is powerful, e.g. controls a large botnet . Dictionary attcker can attempt real popular passwords while delibrately create`popular passwords" to circumvent or relief the constraints imposed by StopGuessing. 





\input{Conclusion}
\section{Acknowledgements}
Jeremiah Blocki was supported in part by NSF CAREER Award \#2047272 and by NSF Award CNS\#1755708. Wuwei Zhang was supported in part by a grant from Purdue Research Foundation. A preliminary draft of this paper was presented at WAY 2019. The authors thank anonymous PC members and shepherd Daniel Votipka for feedback that improved the presentation of this paper.  
\bibliographystyle{ACM-Reference-Format}
\bibliography{../cryptobib/abbrev0,../cryptobib/crypto,../otherReference/CCSStyle}
% !TEX root = main.tex

\begin{theorem}[Hardness of Password Knapsack]\label{appendix:ProofOfPasswordKnapsack}
Find optimal solution for password knapsack is $\NP$-hard.
\end{theorem}

\mypara{Proof:}
We first formally define subset sum problem, and then prove password knapsack is $\NP$ hard by showing the reduction from subset sum to it.
\begin{definition}[Subset Sum]
Given Partition instance $x_1,\ldots,x_{n} \in (0,2^m]$ and target sum value $T$. The  goal is to find $S \subseteq [n]$ s.t. $\sum_{i\in S} x_i = T$? 
\end{definition}
\textbf{Reduction}: One can create the following password knapsack instance 
\begin{itemize}
\item Set $\gamma = \sum_{i=1}^n x_i$,
\item Set $\psi = T/(2\gamma )< \frac{1}{2}$,
\item Set $CS(p_i)= f(p_{i}) = x_i/(2\gamma)$ for $i=1,\ldots, n$
\item Set $f(p_{last}) = 1-\sum_{i =1}^{n} p_i = 1/2 > \psi$. 
\end{itemize}
If $S$ exists for partition instance then attacker can use $S$ for password knapsack to crack $p_{last}+T/(2\gamma)$ passwords. On the other hand let $S$ be the optimal password knapsack solution such that $\sum_{i \in S} CS(p_i) \leq \psi$ then the attacker cracks at most $p_{last}+\sum_{i \in S} f(p_i) \leq 1/2 + \psi$ passwords. If equality holds then $\sum_{i \in S} f(p_i) = \psi$ which implies $\sum_{i \in S} x_i = T$ by definition of $\psi$.


%-----------Above in Appendix----------------
\end{document}
\endinput
%%
%% End of file `sample-sigconf.tex'.
