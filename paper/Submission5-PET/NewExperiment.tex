
\subsection{Results on Subsampling}
We found that it is possible to deploy $\DALock$ without full dataset set based on our subsampling experiments. We remind readers that in such experiment, we implement $\DALock$ with Count Sketch $\sigma_{r\%}$ which is constructed based on sampling r\% of the users from true distriubiton. The results suggest that $\sigma_{r\%}$ can be as effective as $\sigma$ which indicates service providers do not need to collect an enourmous amount of data to implement $\DALock$.

One can observe such effects by comparing curves $\sigma_{1\% dp}$: CS-1\%(k:10,$\Psi:2^{-9.375},\epsilon$:0.1), $\sigma_{5\%}$: CS-5\%(k:10,$\Psi:2^{-9.375},\epsilon$:$\infty$) and $\sigma_{10\%}$: CS-10\%(k:10,$\Psi:2^{-9.375},\epsilon$:$\infty$) with $\sigma$: CS-all(k:10,$\Psi:2^{-9.375}$,$\epsilon:\infty$) from \textbf{Figure}~\ref{figure:dictionaryAttack9.375} and \ref{figure:usability9.375}. (Similarly for $\Psi = 2^{-7.0}$ from \textbf{Figure}~\ref{figure:dictionaryAttack7.0} and ~\ref{figure:usability7.0}.). Our results clearly demonstrates that of $\sigma_{1\%}$, $\sigma_{5\%}$, and $\sigma_{10\%}$ all have comparable performance with respect to $\sigma$ in both security and usability. 

\subsection{Results on Differentially Private Count Sketch}
Consider $\sigma_{1\%dp}$ mentioned in the previous section is a differentially private Count Sketch as well. Notice that for popular password pw, $\TrueFInD{pw}{\SampledData{\AllUser_{1\%}}}$ are roughly 100 times smaller compare to $\TrueFInD{pw}{\SampledData{U}}$ because of subsampling. Similarly, $\TotalFreq{\SampledData{U}} \approx 100\times \TotalFreq{\SampledData{\AllUser_{1\%}}}$. As a result, applying differential privacy to $\sigma_{1\%}$ produce more noisy results. Our experimental results on $\sigma_{1\%dp}$ show that it's performance is close to $\sigma$ in terms of both usability and security. This results suggest that with a reasonably large dataset, it's possible to deploy differentially private $\DALock$ with low privacy budget.

\subsection{Results on Other Estimators}
In this section we demonstrate our findings on implementing $\DALock$ with other estimators. We considered the following password strength meters: $\ZXCVBN$\cite{USENIX:Wheeler16}, Probablistic Context Free Grammar(PCFG), Neural Network, and Hashcat. The later four estimators are provided by Password Guessing Service\cite{USENIX:USBCCKKMMS15}. Since PGS also highly recommends use a hybird approach which takes the min guessing number of the later four apporaches to estimate the strength of passwords, we also include such methodology in our experiments. 

\noindent\textbf{Remark}: We omit the experiment results on Yahoo! for this part as password strength meters estimate the strength of passwords based on their string content. 

We discovered that it is feasible to deploy $\DALock$ with the above estimators; however, deploying them with an inapporiate $\hitCountThreshold$ can yields less demanding results. The experiments results can be found in \textbf{Figure}~\ref{figure:dictionaryAttack9.375}, \ref{figure:usability9.375}, ~\ref{figure:dictionaryAttack7.0}, and ~\ref{figure:usability7.0}. Based on the plots, we found PCFG and $\ZXCVBN$ have satisfying performance by setting $\hitCountThreshold = 2^{-9.375}$. Similarly Min, Markov, and Hashcat outperform 3-strike mechanism on both security and usability when $\hitCountThreshold$ is set to $2^{-7}$. We also found it's not possible to implement the estimator with Neural Netowrk (provided by PGS) alone as it regards most passwords as infrequent ones. 

\subsection{Results on Banning Popular Passwords}
We elaborate our findings on implement $\DALock$ with $\sigma_{-10000}$. Recall that $\sigma_{-10000}$ is constructed from the original dataset by banning top 10,000 passwords. In this scenario, the true distribution is flatend because of banning thus small $\hitCountThreshold$ is required to perform throttling. Our experimental results on $\hitCountThreshold$ = $2^{-11}$ can be found in \textbf{Figure}~\ref{figure:dictionaryAttackPrune} and ~\ref{figure:usabilityPrune}. 

Firstly, we found that dictionary attackers are strongly discouraged simply because of the distribution. By comparing \textbf{Figure}~\ref{figure:dictionaryAttackPrune} and \textbf{Figure}~\ref{figure:dictionaryAttack7.0}, one can observe that the compromised rates are significantly reduced. For example, the attacker can only compromise 1\% of users accounts on $\text{LinkedIn}_{-10000}$ while they can acheive 12\% on the original dataset. 

Secondly, we discoverd that Count-Sketch based $\DALock$ can be deployed on a flat distribution. By comparing curves $CS-all(K:10,\Psi: 2^{-11.0}, \epsilon)$ and 3-strike from \textbf{Figure}~\ref{figure:dictionaryAttackPrune}  and ~\ref{figure:usabilityPrune}. Despite the advantage on security side is minor, Count Sketch based $\DALock$ results in lower unwanted locked out rate. 

Thirdly, we do acknowledge that it is a challenging task to implement estimator with password strength meter under a flat distribution. Based on results, it's impossible to find a $\hitCountThreshold$ value such that one can beats 3-strike for both security and usability. 




