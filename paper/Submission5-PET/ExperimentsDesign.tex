% !TEX root = main.tex
\vspace*{-\baselineskip}
\section{Experimental Design} %Done

We evaluate the performance of $\DALock$ through an extensive battery of empirical simulations. In this section, we describe the modeling choices we made when designing our experiments. To simulate the authentication ecosystem, we need to simulate honest users' behavior, the authentication server running $\DALock$, and an online attacker. 

Briefly, when simulating users, we need to model the distribution over users’ passwords, the distribution over honest login mistakes (e.g., typos or recall errors), and the user's login schedule. When simulating the distribution over users’ passwords, we use multiple empirical datasets to define the underlying password distribution. We use a Poisson arrival process to model the frequency of user login attempts~\cite{AC:BloBluDat13}. Our model for users’ mistakes is informed by recent empirical studies of password typos~\cite{CCS:CWPCR17,SP:CAAJR16} and is augmented to simulate other mistakes, i.e., recall errors.  The key question for simulating an authentication server running $\DALock$ is how the (password) frequency oracle $\EstP{\cdot}$ is implemented. We consider two concrete implementations: password strength models~\cite{ USENIX:Wheeler16,USENIX:USBCCKKMMS15,USENIX:MUSKBCC16} (e.g., $\ZX$, Markov Models, Neural Networks) and (differentially private) count sketches. When simulating the attacker, we consider an untargeted one who knows the distribution over user passwords as well as the $\DALock$ mechanism --- including the frequency oracle $\EstP{\cdot}$. We leave the question of tuning DALock to protect against targeted online attackers~\cite{CCS:WZWYH16} as an important direction for future research. We elaborate on each of these key model components below.  We begin \deleted{by} with an overview of the empirical datasets $\SampledData{\AllUser}$ that we used in our experiments.

\vspace*{-\baselineskip}
\subsection{Experimental Datasets}\label{section:experiment:experiment_dataset} 
\vspace*{-\baselineskip}
In this work, we use multiple real-world password datasets. See Table~\ref{table: datasetsummary} for a summary of each dataset including (1) the total number of unique passwords in the dataset, (2) the total number of user accounts in the dataset, (3) the probability of the most popular password, and (4) the cumulative probability of the top 10 passwords. Except for the differentially private Yahoo! frequency corpus\footnote{Anonymized password histograms representing almost 70 million Yahoo! users who logged into their account during a $48$-hour window in May 2011 \cite{SP:Bonneau12}} , each dataset is the result of a data breach which was collected~\cite{SP:Bonneau12} and publicly released~\cite{NDSS:BloDatBon16} with permission from Yahoo!. We remark that this frequency corpus \textit{does not contain any plaintext passwords}, so we did not use password strength models in our experiments involving the Yahoo! dataset. %The frequency corpus consists of anonymized password histograms representing almost 70 million Yahoo! users who logged into their account during a $48$-hour window in May 2011.
 
%In \lazyref{Section}{section:experimentalresult}, we present the result using all datasets except LinkedIn


%\footnote{The LinkedIn dataset we used is a plaintext password corpus (\textit{partially}) recovered constructed from a leak in 2012. It contains approximately $68$ million cracked passwords, but the actual size of the leak is larger. Furthermore, there is a larger (differentially private) frequency corpus (without plaintext) based on $174+$ million passwords~\cite{harsha2020bicycle} that is publicly available. However, this dataset does not include any plaintext passwords. We chose to use the smaller dataset in our experiments so that we could evaluate with frequency oracles based on password models (e.g., $\ZXCVBN$, PCFGs, Neural Networks).}  and Yahoo. Results on these two datasets can be found in \lazyref{Appendix}{appendix:experimentalResults}.

Each dataset defines an empirical password distribution. In each of our experiments, we assume that this distribution matches the real (unknown) user password distribution from which these datasets were sampled. While the empirical distribution may not precisely match the real one, we stress that our analysis focuses on the most popular passwords in the distribution --- the ones that an attacker will try to guess. Because the datasets are all quite large ( the smallest dataset has over $0.5$ million passwords), standard concentration bounds imply that the true probability of a popular password in the distribution will almost certainly closely match the empirical probability.
\vspace*{-\baselineskip}

\begin{table}[h]
	
	\scalebox{0.90}{
		
		\begin{tabular}{|c|c|c|c|c|}
					
			\hline
					
			Dataset     & Passwords & Accounts & $\TrueP{pw_1}$ & $\TrueP{pw_{1-10}}$ \\ \hline
			
			
			Yahoo    & 33,895,873                     &  69,301,337            & 1.1\%  & 1.9\%                \\ \hline
			
			RockYou  & 14,341,564                & 32,603,388                      & 0.89\%  &2.1\%                   \\ \hline
			
			000webhost  & 10,587,915               & 14,960,642                      &  0.081\%&0.48\% \\ \hline
			
			LinkedIn&  6,840,885              & 68,361,064                    &1.53\% &2.82\%                        \\ \hline
			CSDN  & 4,037,268               & 5,908,494                       & 1.29\%     &3.72\%               \\ \hline
			
			clixsense  & 1,628,297               & 2,195,900 &  0.15\% & 0.7\% \\ \hline
			
			brazzers & 587,934 & 925,614 &0.58\% &1.13\%\\ \hline
			
			bfield  & 416,034& 539,434&  0.48\% & 1.97\% \\ \hline
					
		\end{tabular}	
		
	}
	
	\caption{Summary of dataset}\label{table: datasetsummary}
	
\end{table}

\vspace*{-\baselineskip}


\mypara{Ethics:} The datasets we used contain passwords that were previously stolen and subsequently leaked online. The use of such data raises critical ethical considerations; however, such password lists are already publicly available online, so our use of the data does not exacerbate the prior harm to users. We did not crack any new user passwords. Furthermore, the datasets we use have been cleaned of all identifying information beyond the passwords themselves.  In summary, we believe that our use of the leaked data will not exacerbate prior harm to users, and the lockout mechanism we develop and evaluate may help to protect user passwords in the future.

% Comment Unwanted text for later S&p Submission

% !TEX root = main.tex
\vspace*{-\baselineskip}
\subsection{Modeling Users} \label{section:ExperimentDesign-subsection:SimulateUser}

Our model to simulate honest users' behavior consists of three key components: user password selection, login frequency, and mistake model. 

\vspace*{-\baselineskip}
\subsubsection{Simulating Users’ Password Choices}\label{section:ExperimentDesign-subsection:SimulateUser-subsubsection:SimulatePasswordChoice}
\vspace*{-\baselineskip}
In each simulation, we fix a dataset that is used to simulate user password selection. In particular, a dataset consists of a multiset $\SampledData{\AllUser} = \{pw_1,\cdots,pw_N\}$ of $N$ passwords which can be compressed into pairs $(pw,  \TrueFInD{pw}{\SampledData{\AllUser} })$ where $\TrueFInD{pw}{\SampledData{\AllUser} }$ denotes the number of times the password $pw$ occurs in the dataset $\SampledData{\AllUser}$. Each dataset $\SampledData{\AllUser} $ induces an empirical distribution over users’ passwords where the probability of sampling each password $pw$ is simply $\frac{\TrueFInD{pw}{\SampledData{\AllUser}}}{N}$. 

\mypara{Simulating Password Choices} Each simulated user $u$ in our experiment samples 6 different passwords $pw_{u}^0,\ldots, pw_u^6$ \added{from} the empirical distribution and registers with the first sampled password $pw_u^{0}$. The remaining five passwords $pw_{u}^1,\ldots, pw_u^5$  intuitively represent the user's password for other websites and will be used to simulate recall errors (see \lazyref{Section}{section:ExperimentDesign-subsection:SimulateUser-subsubsection:SimulateUserMistake}).  



Move to appendix We remark that the Yahoo! dataset~\cite{SP:Bonneau12,NDSS:BloDatBon16} only contains frequencies without actual passwords i.e., instead of recording the pair $(pw,  \TrueFInD{pw}{\SampledData{\AllUser} })$ the dataset simply records $\TrueFInD{pw}{\SampledData{\AllUser} }$ . We generate a complete password dataset by designating a unique string for each password. As we avoid using password models like $\ZX$ to analyze $\DALock$ with the Yahoo! dataset since frequency estimation requires access to the original passwords. However, we are still able to evaluate $\DALock$ with the Yahoo! dataset using the Count-Sketch frequency oracle. 

\mypara{Ban-list} We additionally consider the setting where the authentication server chooses to ban users from selecting the top $B$ passwords, e.g., top 10 passwords. We use the normalized probabilities model~\cite{BKPS:ACMEC13} to simulate users' password selections under this restriction. In particular, we use rejection sampling to avoid sampling one of the top $B$ passwords. Equivalently, we can let $\SampledData{\AllUser, B}$ denote the dataset $\SampledData{\AllUser}$ with the $B$ most common passwords removed and sample from the empirical distribution corresponding to the updated dataset $\SampledData{\AllUser, B}$.







%Move to appendix \wuwei{This paragraph needs polishing}

%\mypara{Simulating Password Choices on Yahoo!} Simulating user's choice of password on Yahoo dataset involves an extra step: generate plaintext password string because it only contains the statistics of passwords. Notice that based on \textbf{Table}~\ref{table: datasetsummary},

%Yahoo contains more unique passwords than the rest two; therefore, it is impossible to map the dataset fully. To conquer this issue, we map password strings from RockYou to Yahoo as follows. Firstly, we sample users' choice of passwords from Yahoo based on its current distribution. Secondly, we map the top 20,000 passwords from RockYou to the top 20,000 passwords of Yahoo. For example, $pw_1$ and $pw_2$ from Yahoo are represented by ``123456" and ``12345" respectively, the top 2 passwords from RockYou. The goal of this step is to ensure an adequate string representation of popular passwords. Literatures\cite{EPRINT:WJHW14,TIFS17:WCWPXG,ESORICS:WanWan16,SP:BloHarZho18} suggests that Yahoo and RockYou both follow Zipf's law. Thus there are huge gaps among the top ranks. Thirdly, we map the rest of (RockYou) passwords to Yahoo according to their rankings and the users' choice.  For selected passwords, we map them from passwords with similar ranks in RockYou. On the other hand, we map the unselected passwords uniformly, roughly every 4, from RockYou. 
\vspace*{-\baselineskip}
\vspace*{-\baselineskip}


\subsubsection{Simulating User's Login Patterns}\label{section:ExperimentDesign-subsection:SimulateUser-subsubsection:SimulateLoginPattern} %Done
\vspace*{-\baselineskip}
To simulate users, we need to model the frequency with which our honest user attempts to login to the authentication server. In particular, we aim to simulate the login behaviors over a 180-day time span. For each user $u$, we want to generate a time sequence $0 < t_1^u < t_2^u < \cdots < 4320 = 180\times24$ where each $t_i^u \in \mathbb{N}$ represents the time (in hours) of the $i$th user visit. Following prior works (e.g., see \cite{AC:BloBluDat13,CCS:KogManBon17}), we use a Poisson arrival process to generate the sequence. The Poisson arrival process is parameterized by an arrival rate $T_u$ (hours), which encodes the expected time between consecutive login attempts $T_u = \mathbb{E}[t_{i+1}-t_i]$. The arrival process is memoryless, so the actual gap $t_{i+1}-t_i$  is independent of $t_i$. Since some users are more active than others, we pick a different arrival rate $T_u$ for each user $u$ where each $T_u$ is sampled uniformly at random from $\{ 12, 24, 24 \times 3, 24 \times 7, 24 \times 14, 24 \times 30\}$. The parameter $T_u = 12$ (hours) corresponds to users who login to their accounts twice per day on average, while the parameter $T_u = 24 \times 30$ corresponds to a user who visits the site once per month. We assume that users continue attempting to login for each user visit until they succeed or get locked out. 



\vspace*{-\baselineskip}
\vspace*{-\baselineskip}
\subsubsection{Simulating Users' Mistakes}\label{section:ExperimentDesign-subsection:SimulateUser-subsubsection:SimulateUserMistake} %Done
\vspace*{-\baselineskip}
The last component of our user model is a mechanism to simulate users’ honest mistakes during the authentication process. Our model relies upon recent empirical studies of password typos~\cite{CCS:CWPCR17,SP:CAAJR16} and additionally incorporates other common user mistakes, e.g., recall errors. The aforementioned studies show that roughly $7.5\%$ of login attempts are mistakes, and at least $68\%$ of them are (most likely) typos, i.e., within edit\deleted{ing} distance $2$ of the original passwords.  



Accordingly, in our simulation we set the mistake rate to be $7.5\%$, i.e., when simulating each login attempt, the user will enter the correct password with probability $92.5\%$. Otherwise, we simulate the user's error(s) --- either a recall error or a typo or both. In our simulations of user errors we first flip a biased coin to determine whether to simulate a typo ($68\%$) or a recall error ($32\%$). To simulate a recall error, we randomly select one of the user's five alternate passwords to model a user who forgot which of their passwords was associated with this particular account (the user may additionally misstype this password). When simulating different types of typos (captalization errors, substitution errors, insertion/deletion errors) we rely on empirical password typo data from  \cite{SP:CAAJR16,CCS:CWPCR17}.   We refer an interested reader to \lazyref{Appendix}{appendix:simulateMistakes} for a more detailed discussion of our mistake model, including a flow chart (see Figure~\ref{figure:flowChartTypo}) and more fine-grained typo statistics. If the login attempt is incorrect the simulated user will repeat the above process until s/he is successful or until the account is locked.

%Based on the statistics mentioned earlier, we simulate typos and recall errors with probability $68\%$ and $32\%$, respectively. To simulate a recall error, we randomly select one of the user's five alternate passwords to model a user who forgot which of their passwords was associated with this particular account. If the user recalls the wrong password, they might additionally miss-type it (with probability $0.075\cdot 0.68$).



{\bf Remark: } To study the throttling effects of $\DALock$, we do not simulate users who {\em completely} forget their passwords ( i.e., meaning that the probability of remember\added{ing} the correct password is non-zero during each login attempt) as these users will need to reset their passwords independently of the deployed throttling mechanism. In addition, we do not simulate a client device that automatically attempts to login on the user's behalf using a stored password. It may be desirable to have the authentication server stores the (salted) hash of the user’s previous password(s) to avoid locking the user's account in settings where a client device might repeatedly attempt to login with an outdated password incrementing both the hit-count $\hitCountThresholdOfU{u}$ and the strike count $\KOfU$. Alternatively, the authentication server could store an encrypted cache of failed login attempts using public-key cryptography. Each failed login attempt $pw_u' \neq pw_u$ would be encrypted with a public key $pk_u$ and stored on the authentication server. The encrypted cache could only be decrypted when the user authenticates with the correct password\footnote{Unlike the public encryption key $pk_u$, which would be stored on the authentication server, the secret key $sk_u$ would only be stored in encrypted form i.e., the server would store $c_u = \mathbf{Enc}_{K_u}(sk_u)$ where $K_u = \mathbf{KDF}(pw_u)$ is a symmetric encryption key derived from the user's password. }. The encrypted cache could be used as part of a personalized typo corrector~\cite{CCS:CWPCR17} and could also be used to avoid penalizing repeat mistakes~\cite{CCS:CWPCR17,EuroSP:THS19}. One potential downside to this approach is that the cache might inadvertently contain credentials from other user accounts, making cached data valuable to the attacker. More empirical studies would be needed to determine the risks and benefits of maintaining such a cache.


%Wuwei: Below is mentioned previously.


%We remark that we do not attempt to simulate a user who completely forgets his password. Of course, we expect that this will occasionally happen in reality. However, we observe that a user who forgets his password will {\em always} need to reset it regardless of the throttling mechanism adopted by the authentication server.













% !TEX root = main.tex

\subsection{Modeling the Authentication Server}\label{section:ExperimentDesign-subsection:SimulateServer} %Done

We model an authentication server running $\DALock$ with various parameters $\strikeThreshold$ and $\hitCountThreshold$ for the strike count and hit count. Each time a user $u$ (or attacker pretending to be $u$) attempts to login the authentication server updates the parameters $\hitCountThresholdOfU{u}$ and $\strikeThresholdOfU{u}$ accordingly following the $\DALock$ mechanism. We remark that when $ \hitCountThreshold = \infty$ that the authentication server is running the classical $ \strikeThreshold$-strikes lockout policy. To deploy $\DALock$ with a finite hit-count parameter $ \hitCountThreshold$ an authentication server needs to use a frequency oracle to update the hit count after each incorrect login attempt.  In this work we consider two concrete approaches the authentication server might adopt: (differentially private) Count Sketch estimator and Password Strength Models. We use $\EstimateP{pw}{\Estimator}$ to denote the estimated popularity (probability) of a password $pw$ using the estimator $\Estimator$ e.g., given a Count-Sketch $\sigma$ we would use  $\EstimateP{pw}{\sigma} = \frac{\mathbf{Estimate}(pw,\sigma)}{\mathbf{TotalFreq(\sigma)}}$. We remark that the authentication server might (optionally) chose to ban overly popular passwords to flatten the password distribution to protect user accounts against online attackers \cite{HTS:SchHerMit10}. If the authentication server adopts such policy, then the frequency oracle would need to be adjusted accordingly to model the new password distribution.



\subsubsection{Differentially Private Count Sketch Estimator} 

The first instantiation of $\EstimateP{\cdot}{\cdot} $ we consider is to build a Count Sketch Estimator $\sigma_{\SampledData{\AllUser}} = \Add{\SampledData{\AllUser}}{\sigma} $ from our dataset $\SampledData{\AllUser} $ of user passwords. To build a Count Sketch in practice the authentication server would update the Count Sketch with the new password each time a user registers \footnote{The Count Sketch instantiations we consider would also support a Remove operation which would allow the authentication server to handle password updates efficiently}. There are several issues to consider when deploying the Count Sketch estimator: memory efficiency, privacy, sample size and accuracy. 


\textbf{Memory Efficiency} We instantiate the Count Sketch with parameters $d=5$ and $w=10^6$ so that the entire data structure requires just $20$ MB of space which easily fits in RAM. 


\textbf{Privacy} As we discussed earlier one concern about storing a Count Sketch $\sigma_{\SampledData{\AllUser}} $ on the authentication server is that an offline attacker might steal this file and use the data-structure to help identify user passwords. For example, if our user John Smith selects (resp. does not select) the password ``J.S.UsesStr0ngpwd!'' then we would expect that the true frequency of this password is $\TrueFInD{pw}{\SampledData{\AllUser} }=1$ (resp. $\TrueFInD{pw}{\SampledData{\AllUser} }=0$). If the Count Sketch estimator is overly accurate then the attacker would be able to learn that one user (most likely John Smith) picked this password. Without a way to address these privacy concerns an organization might be understandably wary to deploy a Count Sketch estimator.


To address these privacy concerns we consider an $\epsilon$- differentially private estimator $\sigma_{dp}$ = \textbf{DP($\epsilon,\sigma$)} in our experiments. During initialization we add Laplace noise to each of the cells in the Count Sketch where the noise parameter scales with $d/\epsilon$. In our above example, differential privacy ensures that --- up to a multiplicative advantage $e^{\epsilon}$ --- an attacker cannot use the count sketch to distinguish between a dataset in which John Smith did (resp. did not) pick the password ``J.S.UsesStr0ngpwd!' We remark that lower values of $\epsilon$ correspond to stronger privacy guarantees e.g., we use $\epsilon=\infty$ to denote the case with no differential privacy guarantees. In most of our experiments we use a small privacy parameter $\epsilon=0.1$ which is much smaller than the privacy parameters used in most prior deployments of differential privacy e.g.,  \cite{NDSS:BloDatBon16,AppleDPTeam,CCS:ErlPihKor14}. 


\textbf{Sample Size and Accuracy} In general the accuracy of a Count Sketch increases with the size of the password dataset. Suppose that the organization does not have millions of users or the that the sample size is decreased because the organization allows users to ``opt-in'' to the (differentially private) count sketch. One natural question is whether a smaller organization would be able to deploy a Count Sketch to obtain reliable frequency estimates. We investigate this question by subsampling smaller datasets to train the Count Sketch. Given a set $\AllUser$ of $N$ users we use $\AllUser_{r\%}$ to denote a randomly subsampled set of $r\%$ of users. We use $\SampledData{\AllUser_{r\%}}$ to denote the corresponding subsampled password dataset $\sigma_{r\%} = \Add{\SampledData{\AllUser}}{\sigma} $ to denote the Count Sketch trained on the subsampled data. The question is whether $\sigma_{r\%}$ can be as effective as $\sigma$ for deploying $\DALock$. 


In our experiments we consider the following sampling rates: 1\%, 5\%, and 10\%. We find that even when $r=1\%$ the Count Sketch $\CountSketch$ trained on $\SampledData{\AllUser_{1\%}}$ is sufficiently accurate --- even if we additionally add Laplace noise to preserve $\epsilon=0.1$-differential privacy. 


%Let $\sigma_{r\%}$ be the Count Sketch trained based on $\AllUser_{r\%}$, r\% of the users,  who choose to participate. The question is whether $\sigma_{r\%}$ can be as effective as $\sigma$ for deploying $\DALock$. To investigate it's deployability, we include $\sigma_{r\%}$ constructed by $\AllUser_{r\%}$ as part of our experiments. 


%Count Sketch Estimator $\sigma$ is the first approach for implementing frequency oracle. A standard Count Sketch $\sigma$ trained based on input dataset $\SampledData{\AllUser}$ (as described in \textbf{Section}~\ref{section:Prelinmaries-CountSketch}) can be memory-efficient and accurate. However, $\sigma$ may not be obtainable or usable due to various reasons in real life scenarios (e.g. information protection law). Therefore, we also consider the following types of $\CountSketch$ estimators.


%\textbf{Differential Private Count Sketch}

%Privacy leakage risks exist if one directly deploys $\DALock$ with standard $\sigma$. One sounding solution is to adopt differential privacy to reduce the surface of vulnerabilities; however, it's questionable whether Count-Sketch based $\DALock$ can still be effective when privacy budget is limited. e.g. $\epsilon = 0.1$. To assess its effectiveness under such situation, we also include differential private Count-Sketch estimator $\sigma_{dp}$ = \textbf{DP($\epsilon,\sigma$)} in the experiments. In particular, we are interested in low privacy budget scenarios so data curators can periodically update $\sigma_{dp}$ without significant cumulative privacy loss.




%\textbf{Building Count Sketch with Low Participation Ratio}

%There are multiple incentives to train a Count Sketch on a small dataset. For instance, new business may not have an enormous number of users at beginning. For mature ones, one challenge data curator may face is low participation in password statistics sharing. Human generated passwords can contain sensitive information such data of birth, therefore it's possible that not everyone is willing to opt-in. 




\textbf{Count Sketch with Banlists} In our simulations we also consider an authentication server that bans the most popular $B=10^4$ passwords in a dataset to help flatten the password distribution and protect users against online attacks. Theoretical analysis indicates that directly banning the most popular passwords is the most effective way to increase the minimum entropy of the password distribution~\cite{BKPS:ACMEC13}. We remark that one additional benefit of using a Count Sketch data structure is that it can be used to help implement this type of policy i.e., if a user attempts to register with password $pw$ and $\EstimateP{pw}{\sigma}$ is already too high then the user will be required to pick a different password~\cite{HTS:SchHerMit10}.


We evaluate the performance of $\DALock$ in the presence of banlists. Recall that we let $\SampledData{\AllUser, B}$ denote the dataset $\SampledData{\AllUser}$ with the $B$ most common passwords removed following the normalized probabilities model of ~\cite{BKPS:ACMEC13} to model how affected users will update their passwords in response to the banlist. In particular, we assume users who are affected by the policy will pick a new passwords following the empirical distribution induced by $\SampledData{\AllUser, B}$. We then train the Count Sketch on the updated dataset i.e., $\sigma_{-B} = \text{Add}(\SampledData{\AllUser, B})$ as follows. 


% The core idea behind $\DALock$ is punishing attempts with over popular passwords. One natural following question is what happens if the distribution is not so skewed? One way to achieve such distribution is via banning over popular credential\cite{HTS:SchHerMit10}. Recent studies show\cite{BKPS:ACMEC13,HTS:SchHerMit10} that password composition policies can flaten distributions and discourage attackers. $\DALock$ punishes attempts on over popular credentials while there is no such password under this circumstance. It is worthwhile to test if $\DALock$ can still outperform traditional throttling mechanism. To conduct this type of experiment we implement $\DALock$ with Count Sketch $\sigma_{-b} = \text{Add}(\SampledData{-b})$ as follows. Let b the number of banned passwords, we first construct $\SampledData{-b}$ via banning top b passwords in $\SampledData{\AllUser}$. We assume users who are affected by the policy will pick a new password based on the distribution of remaining passwords in this process. After that, we construct $\sigma_{-b}$ based on the obtained dataset $\SampledData{-b}$



\subsubsection{Frequency Oracle from Password Models}

As we previously discussed there are several reasons why an organization might prefer not to use a Count Sketch for frequency estimation e.g., privacy concerns or limited sample size. An alternative is to instantiate the frequency oracle with a password model. This could be a heuristic password strength meter, a more sophisticated model based on Neural Networks, Probabilistic Context Free Grammars or Markov Models or an empirical estimate based on Hashcat. The primary advantage to this approach is that the model can be deployed immediately even before an organization has any users and there are no privacy concerns. 


We adopted the $\ZXCVBN$ password strength meter~\cite{USENIX:Wheeler16} as prior empirical studies  demonstrate that it is one of the most accurate password strength meters \cite{CCS:GolDur18}. We used the Password Guessing Service \cite{USENIX:USBCCKKMMS15} to obtain guessing numbers for Neural Network, PCFG, Hashcat, and Markov Models ---  we also considered the  minimum guessing number across all four models as suggested in \cite{USENIX:USBCCKKMMS15}. For example, if a password $pw$ had guessing number $g$ we might estimate that $\EstP{pw_i} =1/g$. One challenge that we needed to address was that the estimates we obtain do not always yield a probability distribution e.g., for $\ZXCVBN$ we have $\sum_{i=1}^{10000}\EstP{pw_i} \gg 1$ where $i$ ranges over the top $10^4$ passwords in the dataset. Thus, before deploying the frequency estimator in $\DALock$ we renormalized our estimates so that $\sum_{i=1}^{10000}\EstP{pw_i} =1$. 


% In this work we also seek alternatives to $\CountSketch$ which does not require collecting users' passwords from the server. Literature\cite{CCS:GolDur18} shows that some of them can adequately compute the strength of weak credentials. Therefore, we implement $\DALock$ with the following frequency oracles and tested their performance: $\ZXCVBN$\cite{USENIX:Wheeler16}, Neural Network, PCFG, Hashcat, and Markov Model ( the later four are based on estimation of PGS\cite{USENIX:USBCCKKMMS15}). 

% Implementing frequency oracle by this approach can be a challenging task as they output (strength) scores for passwords in lieu of popularities. To overcome the issue, we define the popularites of password $pw$ to be its estimated score over the total score of top 20,000 passwords. i.e $\EstP{pw} = \frac{\EstP{pw}}{\sum_{i=1}^{20000}\EstP{pw_i}}$.


% !TEX root = main.tex
\subsection{Modeling the Attacker}\label{section:ExperimentDesign-subsection:SimulateAttacker} % Done
The final component of our simulation is a model of the attacker. We take a conservative approach and model an untargetted attacker with complete knowledge of the password distribution. Following Kerckhoff's principle we also assume that the attacker has access to the complete description of the $\DALock$ mechanism. In particular, for any password $pw$ we assume that the attacker knows both the true probability $ \TrueP{pw}$ and the estimated probability $\EstP{pw}$.  Finally, we also assume that the attacker is given the complete sequence of login times $t_1^u \leq t_2^u \leq  \ldots \leq 24 \times 180$ for each user $u$ over a 180 day time span as well as the outcome of each login attempt e.g., at time $t_i^u$ user $u$ will succeed after 2 incorrect guesses. 

{\bf Remark:} We conservatively aim to overestimate the capabilities of an untargetted online attacker. In practice, the online attacker will be able to able to approximate $ \TrueP{pw}$  and $\EstP{pw}$ overtime by interacting with the $\DALock$ server e.g., by setting up dummy accounts to test many times he can submit a particular incorrect guess without exceeding the hit count. Similarly, the attacker would not necessarily know the exact login times for a user, but this conservative assumption makes it feasible to precisely characterize the optimal behavior of an attacker. In practice, an online attacker might wait several days in between guesses to avoid accidently locking the user's account based on the number of consecutive incorrect login attempts. 

\subsubsection{Optimizing Attack Strategies} %Done
The goal of the attacker is to maximize the probability of cracking each password within the fixed 180-day time span. For example, the attacker might try to find popular passwords $pw$ where the ratio $\EstP{pw}/\TrueP{pw}$ is small so that the increased hit count is smaller than intended. We formalize the attacker's optimal strategy in terms of the \textsf{Password Knapsack} problem $(\PK)$. Unsurprisingly, the password knapsack problem turns out to be $\NP$-hard (as we prove in the appendix), but there are several heuristic algorithms the $\Adversary$ can use which yield nearly optimal strategies in practice. 

Recall that we assume that the $\Adversary$ has perfect knowledge of the distribution and probability estimates for each password $pw$. We also assume $\Adversary$ knows the $\DALock$ security parameters $\strikeThreshold$ and $\hitCountThreshold$. Furthermore, for each user $u$ we assume that the attacker is given the complete sequence of login times $t_1^u \leq t_2^u \leq  \ldots \leq 24 \times 180$ for each user $u$ over a 180 day time span as well as the outcome of each login attempt e.g., at time $t_i^u$ user $u$ will succeed after 2 incorrect guesses. In particular, at any point in time $t < 24\times 180$ the attacker can infer the current strike threshold and hit count threshold for any user $u$. We denote by $\strikeThresholdOfU{u,t}$ (resp. $\hitCountThresholdOfU{u,t}$) the strike (resp. hit count) threshold  for user $u$ at time $t$ assuming that the attacker does not submit any of his own guesses. 

Supposing that the attacker wishes to avoid locking down the user's account before time $t$ the cumulative (estimated) probability of all guesses submitted before that time should be at most $\hitCountThresholdOfU{u,t}':=\hitCountThreshold- \hitCountThresholdOfU{u,t}$. Similarly, we let $M(t)$ denote the maximum number of guesses that the attacker can sneak in over the first $t$ hours without locking down the account i.e., because $\strikeThresholdOfU{u,t'}  \geq \strikeThreshold$ at some time $t' \leq t$. 

Fixing the time parameter $t$ the attacker’s goal is to find a subset $S_t \subseteq \AllPassword$ of $M(t)$ passwords to check such that 
\begin{equation} \label{eq:attackerConstraint}
	\vspace{-0.2cm} 
	\sum_{pw \in S_t} \EstP{pw} \leq \hitCountThresholdOfU{u,t}' \ . \vspace{-0.1cm} 
\end{equation}
After checking the passwords in $S_t$ the attacker can still check one more password $pw_{hold} \not\in S_t$ before the account is locked down. Given a set $S_t$ and a holdout password $pw_{hold} \not\in S_t$ the probability that the attacker succeeds is 
\begin{equation}\vspace{-0.2cm} \label{eq:attackerSuccess} \TrueP{pw_{hold}} + \sum_{pw \in S_t}\TrueP{pw} \ . \vspace{-0.1cm} \end{equation}

Thus, the goal of the attacker is to find a subset $S_t$ of size $|S_t| \leq M(t)$ maximizing his success rate (eq \ref{eq:attackerSuccess}) subject to the constaint in  equation \ref{eq:attackerConstraint}.

\mypara{Password Knapsack Problem}  Given a password dictionary \\$\{pw_1, \ldots, pw_n\}$ we formally define the \textsf{P}assword \textsf{K}napsack($\PK$) problem as the following integer program with indicator variables $s_i \in \{0,1\}$ and $l_i=\{0,1\}$ for each password $pw_i$. The attackers goal is to select a holdout password and a separate subset of $M$ ($=M(t)$) passwords with total `weight' (estimated probability) at most $\hitCountThreshold'$ ($= \hitCountThresholdOfU{u,t}'$) 

$$
\begin{array}{crl}
	&\max {\displaystyle{\sum_i {(s_i + l_i) \cdot \TrueP{pw_{i}}}}} \\
	subject\ to, &\\
	&\sum_i{s_i \cdot \EstimateP{pw_i}{\sigma}) \le \hitCountThreshold'} \\
	&\sum_i s_i \le M\\
	&\sum_i l_i \le 1\\
	&\forall i~ l_i + s_i \le 1\\
	where,\\
	& \forall i, s_i, l_i \in \{0,1\}
\end{array}
$$
Intutively, setting $s_i$ = 1 means $pw_i$ is selected to be placed in the ``password knapsack" $S\subseteq \AllPassword$, i.e. to be used for dictionary attack. Setting $l_i=1$ indicates that password $pw_i$ is used as holdout password. This is equivalent to the following optimization problem. The constraints ensure that $|S| \leq M$ and we pick exactly one holdout password that is not already in $S$. 

\mypara{Solving the \textsf{P}assword \textsf{K}napsack} To maximize the number of cracked passwords an online attacker can compute $M(t)$ and $\hitCountThresholdOfU{u,t}':=\hitCountThreshold- \hitCountThresholdOfU{u,t}$ for each time $t \leq 24 \times 180$ and solve the corresponding \textsf{P}assword \textsf{K}napsack problem. Given optimal solutions $(pw_{hold,t}^*, S_t^*)$ for each time $t$ the attacker will pick the solution that maximizes the number of cracked passwords as in equation \ref{eq:attackerSuccess}. We remark that the calculations above need to be repeated for each different user $u$ since the values $M(t)$ and $\hitCountThresholdOfU{u,t}'$ may vary due to different visitation schedules.







\mypara{Solving Password Knapsack}  Unfortunately, the \textsf{P}assword \textsf{K}napsack problem is $\NP$-hard as we prove in \textbf{Theorem}~\ref{appendix:ProofOfPasswordKnapsack} in the Appendix via a straightforward reduction from Subset Sum. In all of instances we considered we found that the optimal choice for the holdout password was simply $pw_1$ the most likely password in the distribution. Once we fix our holdout password our problem reduces to the two-dimensional knapsack problem. We remark that $\PK$ can be viewed as a two-dimensional knapsack problem. 

Assuming $P\neq NP$ the two-dimensional knapsack problem does not even admit a polynomial time approximation scheme ($\PTAS$) \cite{kulik2010there} in contrast to the regular knapsack problem which has fully polynomial time approximation scheme ($\FPTAS$)). Thus, we consider two heuristic approaches to solve the password knapsack problem:  $\mathsf{D}$antizig's $\mathsf{A}$lgorithm $\mathsf{B}$ased\cite{Dan:OR57} approach (\DAB) and $\mathsf{F}$easible $\mathsf{M}$ost $\mathsf{P}$romising $\mathsf{P}$assword $\mathsf{F}$irst approach(\FMPPF).

$\DAB$ (\textbf{Algorithm}~\ref{algorithm:Dantizig}, Appendix) sorts the remaining passwords $\mathcal{P}_{\tilde{\Pi}} = \{pw_2, \ldots\, pw_n\}$ based on the ratios $\frac{\TrueP{pw_i}}{\EstP{pw_i}}$ and select candidates based on the sorted order until we either select $M$ passwords or until selecting another password would exceed our capacity $\hitCountThreshold'$. $\FMPPF$ (\textbf{Algorithm}~\ref{algorithm:FMPPF}, Appendix) sorts the remaining passwords based on the true probability $\TrueP{pw_i}$  and simply selects password $pw$ in sorted order until we either select $M$ passwords or until selecting another password would exceed our capacity $\hitCountThreshold'$.  We discuss the advantages and disadvantages to both heuristics in the appendix. Intuitively, $\FMPPF$ (resp. $\DAB$) will perform better when $M$ (resp. $\hitCountThreshold'$) is the limiting constraint.

We found that $\FMPPF$ generally performs better than $\DAB$ despite of its simplicity. In addition, our simuation shows that $\FMPPF$'s performance is close to optimal. Practically speaking, one generally expect $\EstP{pw_i} \approx \TrueP{pw_i}$ especially when $pw_i$ is a popular password. In such case, $\DAB$ can hardly gain advantages from underestimation. Furthermore, imagine one bucket passwords by probability ranges, there are plenty of passwords in each bucket. Intuitively, picking passwords ordered by $\TrueP{pw_i}$ should produce an (almost) optimal solution (quickly). Thus, we choose to present the results based on $\FMPPF$ approach.





