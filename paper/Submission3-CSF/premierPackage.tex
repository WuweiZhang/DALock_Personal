% !TEX root = main.tex

% Packages All the Way

\usepackage{balance,color,amsmath, alltt, xspace, epsfig, algorithm, subfigure, xcolor, multirow, psfrag,mathtools,comment, verbatim,algpseudocode,grffile}					
\usepackage{xurl}

%\usepackage[latin1]{inputenc}
\usepackage{tikz}
\usetikzlibrary{shapes,arrows}

\tikzstyle{decision} = [diamond, draw, fill=blue!20, 
text width=4.5em, text badly centered, node distance=3cm, inner sep=0pt]
\tikzstyle{block} = [rectangle, draw, fill=blue!20, 
text width=5em, text centered, rounded corners]
\tikzstyle{line} = [draw, -latex']

\tikzstyle{cloud} = [draw, ellipse,fill=red!20, node distance=3cm,
minimum height=2em]

 \usepackage{graphicx} 
\usepackage{caption}
\captionsetup[figure]{font=small}
%\usepackage[hyphens]{url}
\usepackage{hyperref}
\hypersetup{breaklinks=true}


% Variable Part
\newcommand{\algoname}{DALock}					%Give a name to the algorithm/theorem


\newcommand{\authnote}[3]{\textcolor{#2}{{\sf (#1's Note: {\sl{#3}})}}}
\newcommand{\wuwei}{\authnote{Wuwei}{green}}
\newcommand{\jeremiah}{\authnote{Jeremiah}{blue}}
\newcommand{\ignore}[1]{}

% Fixed/Constant Part:
%\newtheorem{theorem}{Theorem}
%\newtheorem{definition}{Definition}
\newcommand{\mypara}[1]{\noindent\textbf{#1} \xspace}		% Bold paragraph title
\newcommand{\says}[2]{{\color{blue}{#1 says: }{#2}}\xspace}					% Says 
\DeclarePairedDelimiter{\ceil}{\lceil}{\rceil}								%\ceil	

%---------- Macros for Table (Below)-------------
\newcommand{\PasswordOfU}[1]{pw_{#1}}
\newcommand{\PwOfU}[1]{pw_{#1}}
\newcommand{\PsiOfU}{\Psi_u}
\newcommand{\KOfU}{K_u}
\newcommand{\PwProbEstimator}{\mathsf{Est}}
\newcommand{\AllUser}{\mathcal{U}}
\newcommand{\RankRPassword}[1]{pw_{#1}}
\newcommand{\user}{u}
\newcommand{\Lap}{\mathsf{Lap}\xspace}								% Laplace Noise
\newcommand{\epsLap}[2]{\ensuremath{\mathsf{LAP(\frac{#2}{#1})}}\xspace} %Laplace Noise with #1 privacy budget (\epsilon) and #2 sensitivity
\newcommand{\Adversary}{\ensuremath{\mathcal{A}}}							% Lazy \mathcal{A}
\newcommand{\lazyref}[2]{\textbf{#1}~\ref{#2}}
\newcommand{\ZX}{\ZXCVBN} 			%Lazy fancy ZXCVBN
\newcommand{\loginActivity}[3]{\mathsf{L_{#1}(#2,#3)}}
\newcommand{\ZXCVBN}{\mathsf{ZXCVBN}}
\newcommand{\PCFG}{\mathsf{PCFG}}
\newcommand{\Min}{\mathsf{Min}}
\newcommand{\HashCat}{\mathsf{HashCat}}
\newcommand{\NeuralNet}{\mathsf{NeuralNet}}
\newcommand{\Markov}{\mathsf{Markov}}
\newcommand{\Estimator}{\mathsf{Estimator}}
\newcommand{\EstF}[1]{\textsf{Estimate}(#1)}
\newcommand{\EstimateF}[2]{\textsf{Estimate}(#1,#2)}			%Estimate Frequency of p 
\newcommand{\EstP}[1]{\textsf{p}(#1)}
\newcommand{\EstimateP}[2]{\textsf{p}(#1,#2)}			%Estimate popularity of p 
\newcommand{\MM}{\ensuremath{\mathcal{M}}}							% Lazy \mathcal{A}
\newcommand{\FMPPF}{\ensuremath{\mathsf{FMPPF}}}							% Lazy \mathsf{FMPPF}
\newcommand{\DAB}{\ensuremath{\mathsf{DAB}}}	
\newcommand{\PK}{\ensuremath{\mathsf{PK}}}							% Lazy Password Knapsack
\newcommand{\SampledData}[1]{\mathcal{D}_{{#1}}}
\renewcommand{\Pr}[1]{\ensuremath{\mathsf{Pr} \left[#1\right] }\xspace}			% Fancy Pr
\newcommand{\KPsiDALock}[2]{\ensuremath{(#1, #2)\text{-}\mathsf{DALock}}\xspace}		
\newcommand{\hitCountThreshold}{\Psi}
\newcommand{\hitCountThresholdOfU}[1]{\hitCountThreshold_{#1}}	
\newcommand{\DP}[2]{\textsf{DP}(#1,#2)}
\newcommand{\strikeThreshold}{K}	
\newcommand{\strikeThresholdOfU}[1]{\strikeThreshold_{#1}}
\newcommand{\DALock}{\mathsf{DALock}\xspace}	
\newcommand{\AllPassword}{\mathcal{P}}
\newcommand{\CountSketch}{\mathsf{CS}}
\newcommand{\CountSketchCounter}{\mathsf{CS.T}}
\newcommand{\CountSketchArray}{\mathsf{CS.ARRAY}}
\newcommand{\EstProbOfPw}[2]{\ensuremath{\Estimator_{\mathsf{#2}}}\left(#1\right)}
\newcommand{\TrueP}[1]{\ensuremath{\mathsf{P}\left(#1\right)}}
\newcommand{\TrueF}[1]{\ensuremath{\mathsf{F}\left(#1\right)}}
\newcommand{\TrueFInD}[2]{\ensuremath{\mathsf{F}\left(#1, #2\right)}}
\newcommand{\CSWidth}{w}
\newcommand{\CSDepth}{d}
\newcommand{\TotalFreq}[1]{\ensuremath{\mathsf{TotalFreq}(#1)}}
\newcommand{\Add}[2]{\ensuremath{\mathsf{Add}(#1,#2)}}
\newcommand{\Initialize}[2]{\ensuremath{\mathsf{Initialize}(#1,#2)}}
\newcommand{\HashFunRowD}{\ensuremath{\mathsf{h}_d}}
\newcommand{\HashFunSign}{\ensuremath{\mathsf{h}_{\pm}}}
\newcommand{\hitCountThresholdofUAtT}[2]{\hitCountThresholdOfU{#1}^{#2}}
\newcommand{\strikeThresholdofUAtT}[2]{\strikeThresholdOfU{#1}^{#2}}
%---------- Macros for Table (Above)-------------


\newtheorem{definition}{Definition}
\newtheorem{thm}{Theorem}[section]
\newcommand{\NP}{\mathsf{NP}\xspace}

\newcommand{\FPTAS}{\mathsf{FPTAS}\xspace}
\newcommand{\PTAS}{\mathsf{PTAS}\xspace}
\renewcommand{\P}{\mathsf{P}\xspace}	
\newcommand{\SAT}{\mathsf{SAT}\xspace}	
\newcommand{\TIME}{\mathsf{TIME}\xspace}
\newcommand{\NPC}{\mathsf{NPC}\xspace}	
\newcommand{\myexp}[1]{\ensuremath{e^{#1}}\xspace}						% exp


%---------- Macros for General Setting-------------
\algrenewcommand\algorithmicindent{1.0em}%
\usepackage{tikz}
\usetikzlibrary{matrix,calc,shapes}
\tikzset{
  treenode/.style = {shape=rectangle, rounded corners,
                     draw, anchor=center,
                     text width=5em, align=center,
                     top color=white, bottom color=blue!20,
                     inner sep=1ex},
  decision/.style = {treenode, diamond, inner sep=0pt},
  root/.style     = {treenode, font=\Large, bottom color=red!30},
  env/.style      = {treenode, font=\ttfamily\normalsize},
  finish/.style   = {root, bottom color=green!40},
  dummy/.style    = {circle,draw}
}
\newcommand{\yes}{edge node [above] {yes}}
\newcommand{\no}{edge  node [left]  {no}}



