% !TEX root = main.tex

\begin{theorem}[Hardness of Password Knapsack]\label{appendix:ProofOfPasswordKnapsack}
Find optimal solution for password knapsack is $\NP$-hard.
\end{theorem}

\mypara{Proof:}
We first formally define subset sum problem, and then prove password knapsack is $\NP$ hard by showing the reduction from subset sum to it.
\begin{definition}[Subset Sum]
Given Partition instance $x_1,\ldots,x_{n} \in (0,2^m]$ and target sum value $T$. The  goal is to find $S \subseteq [n]$ s.t. $\sum_{i\in S} x_i = T$? 
\end{definition}
\textbf{Reduction}: One can create the following password knapsack instance 
\begin{itemize}
\item Set $\gamma = \sum_{i=1}^n x_i$,
\item Set $\psi = T/(2\gamma )< \frac{1}{2}$,
\item Set $CS(p_i)= f(p_{i}) = x_i/(2\gamma)$ for $i=1,\ldots, n$
\item Set $f(p_{last}) = 1-\sum_{i =1}^{n} p_i = 1/2 > \psi$. 
\end{itemize}
If $S$ exists for partition instance then attacker can use $S$ for password knapsack to crack $p_{last}+T/(2\gamma)$ passwords. On the other hand let $S$ be the optimal password knapsack solution such that $\sum_{i \in S} CS(p_i) \leq \psi$ then the attacker cracks at most $p_{last}+\sum_{i \in S} f(p_i) \leq 1/2 + \psi$ passwords. If equality holds then $\sum_{i \in S} f(p_i) = \psi$ which implies $\sum_{i \in S} x_i = T$ by definition of $\psi$.


%-----------Above in Appendix----------------