% !TEX root = main.tex
\clearpage
\appendix
\section{$\DALock$}



\subsection{$\DALock$ Authentication Algorithm}
We supplement the pseudo code of $\DALock$ in this section to help readers understand how to implement $\DALock$ for authentication. The authentication process takes four arguments: username $u$, input password $pw$, salt $s_u$, and password popularity estimator $\sigma$. Before verifying the correctness of entered password, $\DALock$ first check if $u$'s account has already been locked or not based on $\hitCountThreshold_{u}$ and $\strikeThreshold{u}$. If the account is not locked, $\DALock$ proceeds and verify the correctness of the passwords. If the password is valid, $\DALock$ resets strike threshold $\strikeThresholdOfU{u}$ and grant user the access to the service. If the entered password is wrong then in addition to denying the access to the service, the server also increases $\hitCountThresholdOfU{u}$ and $\strikeThreshold$ by $\EstP{pw}$ and 1 respectively. 


\begin{algorithm}[!htb]
	\caption{\textbf{$\DALock$}: Novel Password Distribution Aware Throttling Mechanism }\label{algorithm:DALock}
	\begin{algorithmic}[1]
		\Function{login}{$u$, $pw_u$, $\sigma$,$s_u$} 
		\If {$\hitCountThresholdOfU{u} \geq \hitCountThreshold$ or  $\strikeThresholdOfU{u} \geq \strikeThreshold$ }
		\State Reject Login
		\EndIf
		\If{$hash(pw,s_u)$ == $hash(pw_u,s_u)$}
		\State Reset $\strikeThresholdOfU{u}$ to $\strikeThreshold$ 
		\State Grant Access
		\Else
		\State $\hitCountThresholdOfU{u} \leftarrow \hitCountThresholdOfU{u} + \EstimateP{pw}{\sigma}$
		\State $\strikeThresholdOfU{u} \leftarrow \strikeThresholdOfU{u}$ + 1
		\State Deny Access
		\EndIf
		\EndFunction
	\end{algorithmic}
\end{algorithm} 
\subsection{Password Knapsack is $\NP$ hard}




\begin{theorem}[Hardness of Password Knapsack]\label{appendix:ProofOfPasswordKnapsack}
	Find optimal solution for password knapsack is $\NP$-hard.
\end{theorem}

\mypara{Proof:}
We first formally define subset sum problem, and then prove password knapsack is $\NP$ hard by showing the reduction from subset sum to it.
\begin{definition}[Subset Sum]
	Given Partition instance $x_1,\ldots,x_{n} \in (0,2^m]$ and target sum value $T$. The goal is to find $S \subseteq [n]$ s.t. $\sum_{i\in S} x_i = T$? 
\end{definition}
\textbf{Reduction}: One can create the following password knapsack instance 
\begin{itemize}
	\item Set $\gamma = \sum_{i=1}^n x_i$,
	\item Set $\psi = T/(2\gamma )< \frac{1}{2}$,
	\item Set $CS(p_i)= f(p_{i}) = x_i/(2\gamma)$ for $i=1,\ldots, n$
	\item Set $f(p_{last}) = 1-\sum_{i =1}^{n} p_i = 1/2 > \psi$. 
\end{itemize}
If $S$ exists for partition instance then attacker can use $S$ for password knapsack to crack $p_{last}+T/(2\gamma)$ passwords. On the other hand let $S$ be the optimal password knapsack solution such that $\sum_{i \in S} CS(p_i) \leq \psi$ then the attacker cracks at most $p_{last}+\sum_{i \in S} f(p_i) \leq 1/2 + \psi$ passwords. If equality holds then $\sum_{i \in S} f(p_i) = \psi$ which implies $\sum_{i \in S} x_i = T$ by definition of $\psi$.





\subsection{Solving $\PK$ with Heruistics}
In this section we supplement the missing details of algorithm $\DAB$ and $\FMPPF$ mentioned in \textbf{Section}~\ref{section:ExperimentDesign-subsection:SimulateAttacker}.  

The $\DAB$ approach takes three inputs: a sorted password dictionary based on the ratio of actual popularity and estimated popularity $\frac{\EstP{pw}}{\TrueP{pw}}$: $\mathcal{P}_{\tilde{\Pi}} = \{pw_{\tilde{\Pi}(1)}, \ldots, pw_{\tilde{\Pi}(n)} \}$, attack budget $\Psi$ and K. The algorithm keeps placing passwords into the knapsack S based on the sorted order until it cannot further add some password $pw$. At this points, $\DAB$ compares $\TrueP{pw}$ with $\TrueP{S}$ and sets S to be the one with higher values. After that, the algorithm repeats the above process by scanning through the whole dictionary. At the end, since only $K$ passwords is allowed to be used for guessing, the algorithm returns $K$ passwords based on their actual probability


Primary incentives of using this algorithm are 1) to take advantage of underestimated passwords and 2) to avoid (severely) overestimated ones. There are several drawbacks of $\DAB$. Firstly, the progress can be slow because priority is given to significantly underestimated passwords. Intuitively, for popular passwords ($\TrueP{pw}$ large) the ratio $\frac{\TrueP{pw}}{\EstP{pw}}$ is likely to be close to 1, therefore, attempts with popular ones are likely to be delayed. Secondly, unlike vanilla version of Knaspack, $\DAB$ may not yield a 2-approximation due to the additional constraint on the number of passwords one can place in the Knapsack. Third, computation cost is slightly higher for running $\DAB$ though both algorithms terminate reasonably quickly. %Thirdly, computation cost can still be a concern for large scale attacks. 



\begin{algorithm}[!htb]
	\caption{\textbf{$\DAB$ Attack }}\label{algorithm:Dantizig}
	\textbf{Return:}  An array of password sorted in the order of guessing
	\begin{algorithmic}[1]
		\Function{$\DAB$ Attack}{$\mathcal{P}_{\tilde{\Pi}}, \Psi, M(T)$}
		\State $S$= []
		\While{$S$ changes}
		\For{ $pw \in \mathcal{P}_{\tilde{\Pi}}$  }
		\If{$\EstP{pw}$ $> \Psi$ }
		\State \textbf{continue}
		\EndIf
		\If{$\EstP{S\cup pw}$ $< \Psi$ and $|S| \le$ $M(T)$  }
		\State $S \leftarrow S \cup pw$ 
		\ElsIf{$\TrueP{pw} > \TrueP{S}$ }
		\State $S \leftarrow$ \{p\}
		\EndIf
		\EndFor
		\EndWhile
		\State \Return Top $M(T)$ passwords from $S$ based on actual popularity.
		\EndFunction
	\end{algorithmic}
\end{algorithm}  





An alternative to $\DAB$ is $\FMPPF$. It takes three input parameters $\mathcal{P} = pw_1, \ldots, pw_{n} \}$, attack budget $\Psi$ and $M(T)$ and selects passwords greedily. $\FMPPF$ differs from $\DAB$ in the following two aspects. Firstly, $\FMPPF$ uses a password dictioary sorted based on the actual popularity only, which can be easily obtained in reaf life. Secondly, to save computational cost $\FMPPF$  terminates once it finds $K$ passwords that are suitable for attacks and stop further explore the dictionary. The pseudo can be found in \textbf{Algorithm}~\ref{algorithm:FMPPF}

In short time attack scenarios, $\FMPPF$ offers better chance of success than $\DAB$ by attempting popular ones first. For long term case, $\FMPPF$ should still be able to achieve almost optimal results given an abundant choice of passwords.  In fact, based on the empirical results (in \textbf{section}~\ref{section:ExperimentResult-security}), the performance of $\FMPPF$ is very close to theoretical upper bounds ($\Psi + \TrueP{pw_1}$ ). %Last but not least, $\FMPPF$ is computational efficient comparing to $\DAB$.


\begin{algorithm}[!htb]
	\caption{\textbf{$\FMPPF$ Approach}}\label{algorithm:FMPPF}
	\textbf{Return:} An array of password sorted in the order of guessing
	\begin{algorithmic}[1]
		\Function{\FMPPF}{$\mathcal{P}, \Psi, M(T)$}
		\State $S$= []
		\For{ $pw \in \mathcal{P}$  }
		\If{$\EstP{S \cup p}$ $< \Psi$ and $|S| < M(T)$  }
		\State $S \leftarrow S \cup pw$ 
		\EndIf
		\EndFor
		\State \Return S
		\EndFunction
	\end{algorithmic}
\end{algorithm} 




\section{Simulating User's Mistakes}
In this section we elaborate the details for simulating users' mistakes missing in \textbf{Section}~\ref{section:ExperimentDesign-subsection:SimulateUser}. To help reader visualize the process of simulating mistakes, we plot the flowchart in \textbf{figure}~\ref{figure:flowChartTypo}. The starting point is to simulate recall errors. Based on the empirical results of existing literatures\cite{CCS:CWPCR17,SP:CAAJR16}, we set the probability of makeing an recall error to 2.4\%. Recall that when we generate user's profile, each user has five ``passwords from other services". Therefore, we simulate recall error by choose one of them to be the \emph{password user intends to enter}. On top of this process, we further simulate typos (on the password intended to enter) with probability $\approx 5\%$. Condition on making typos, we simulate this step by choosing a type of typos with their conditional probability summarized in \textbf{Table}~\ref{Table:TypoTypes}.

\begin{figure}[htb]
	\begin{center}
		\includegraphics[height=2in,width=\linewidth]{Figures/Experiments/FlowChartTypo.png}
		\caption{Flow Chart for Simulating Users' mistake}\label{figure:flowChartTypo}
	\end{center}
\end{figure}


\begin{table}[h]
	\begin{tabular}{|c|c|}
		\hline
		Typo Types           & Chance of Mistake(Rounded \%) \\ \hline
		CapLock On           & 14                            \\ \hline
		Shift First Char     & 4                             \\ \hline
		One Extra Insertion  & 12                            \\ \hline
		One Extra Deletion   & 12                            \\ \hline
		One Char Replacement & 31                            \\ \hline
		Transposition        & 4                             \\ \hline
		Two Deletions         & 3                             \\ \hline
		Two Insertions        & 3                             \\ \hline
		Two Replacements      & 10                            \\ \hline
		Others               & 8                             \\ \hline
	\end{tabular}
	\caption{Typo Distributions\cite{CCS:CWPCR17}}
	\label{Table:TypoTypes}
	\vspace{-1cm}
\end{table}	


\section{More Experimental Results}\label{appendix:experimentalResults}
We provide more experimental results for curious readers to better understand the performance of $\DALock$ in multiple scenarios. To clearly demonstrates the impact of $\hitCountThreshold$ on security and usability we choose to visualize the following set of parameters $\{2^{-6}, 2^{-7},2^{-8},2^{-9}, 2^{-10}\}$(\textbf{Figure}~\ref{figure:appendix_attacker} and ~\ref{figure:appendix_usability}). When one set $\hitCountThreshold$ to an over conservative, e.g. $2^{-10}$ or smaller, dictionary attackers are not able to crack a significant portion of users accounst; however, usebility can be a concern as infrequent passwords burns $\hitCountThreshold_{u}$ quickly.(see curve CS-all($k:10,\hitCountThreshold:2^{-10},\epsilon$) across all figures). Another extreme approach is adopting aggresively large $\hitCountThreshold$, e.g. $2^{-6}$. Based on the plots (curve CS-all($k:10,\hitCountThreshold:2^{-6},\epsilon$)), the usability performance is satisfactory while security risk is enlarged (but still better than 3-strike). Beyond standard Count Sketch, we supplement more results on applying differential privacy to Count-Sketch(\textbf{Figure}~\ref{figure:appendix_attacker_dp} and  ~\ref{figure:appendix_usability_dp}). We discovered that differential privacy hardly has any negative impact on the performance of $\DALock$ especially when the dataset is large. Furthermore, we supplement more results on subsampling to show $\sigma_{r\%}$ is effective for a wide range of $\hitCountThreshold$ even when the sampling rate is 1\%.(\textbf{Figure}~\ref{figure:appendix_attacker_sample} and  ~\ref{figure:appendix_usability_sample}) Finally, we present the results of differentially private $\DALock$ trained on subsampling dataset $\SampledData{U_{1\%}}$. Despite the datasets become 100 times smaller, $\DP{\sigma_{1\%}}$ still performs reasonly well.(\textbf{Figure}~\ref{figure:appendix_attacker_dpandsample} and ~\ref{figure:appendix_usability_dpandsample}).

\mypara{Almost Optimal Heruistic}  If one assumes that $\TrueP{pw} \approx \EstP{pw}$ for popular passwords, then theoretically $\Adversary$ can compromise at most $\hitCountThreshold + \TrueP{pw_1}$ users accounts by using $pw_1$ as holdout passwords. Based on our simulation results, we found that $\Adversary$ is very close to achieve such threshold by adopting $\FMPPF$ algorithm mentioned in \textbf{Section}~\ref{section:ExperimentDesign-subsection:SimulateAttacker}. To help readers identify this property, we highlight the upperbounds(e.g. $2^{-6} + \TrueP{pw_1}$ etc)

\clearpage
\begin{figure*}[htb]
	\includegraphics[width=\linewidth,height=4cm]{Figures/Experiments/Attacker/Security_Appendix.png}
	\caption{Security: $\sigma$}\label{figure:appendix_attacker}
	\includegraphics[width=\linewidth,height=4cm]{Figures/Experiments/Utility/Usability_Appendix.png}
	\caption{Usability:  $\sigma$}\label{figure:appendix_usability}
	\includegraphics[width=\linewidth,height=4cm]{Figures/Experiments/Attacker/Security_Appendix_dp.png}
	\caption{Security: $\DP{0.1}{\sigma}$}\label{figure:appendix_attacker_dp}
	\includegraphics[width=\linewidth,height=4cm]{Figures/Experiments/Utility/Usability_Appendix_dp.png} 
	\caption{Usability: $\DP{0.1}{\sigma}$}\label{figure:appendix_usability_dp}
\end{figure*}
\begin{figure*}[htb]
	\includegraphics[width=\linewidth,height=4cm]{Figures/Experiments/Attacker/Security_Appendix_1p.png}
	\caption{Security: $\sigma_{1\%}$}\label{figure:appendix_attacker_sample}
	\includegraphics[width=\linewidth,height=4cm]{Figures/Experiments/Utility/Usability_Appendix_1p.png}
	\caption{Usability: $\sigma_{1\%}$}\label{figure:appendix_usability_sample}
	\includegraphics[width=\linewidth,height=4cm]{Figures/Experiments/Attacker/Security_Appendix_dp_1p.png}
	\caption{Security: $\DP{0.1}{\sigma_{1\%}}$ }\label{figure:appendix_attacker_dpandsample}
	\includegraphics[width=\linewidth,height=4cm]{Figures/Experiments/Utility/Usability_Appendix_dp_1p.png} 
	\caption{Usability:$\DP{0.1}{\sigma_{1\%}}$}\label{figure:appendix_usability_dpandsample}
\end{figure*}
\clearpage


